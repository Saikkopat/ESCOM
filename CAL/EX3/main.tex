\documentclass[a4paper,12pt]{report}
\usepackage{geometry}
\usepackage{graphicx}
\usepackage{tikz}
\usepackage{float}
\usepackage[document]{ragged2e}
\usepackage[utf8]{inputenc}
\usepackage[T1]{fontenc}
\usepackage[spanish]{babel}
\renewcommand{\shorthandsspanish}{}
\usepackage{amsmath}
\usepackage{pgfplots}
\usepackage{xparse}
\usepackage{dsfont}

\graphicspath{ {/home/saikkopat/Documents/school/CAL/EX1/} }

\NewDocumentCommand{\INTERVALINNARDS}{ m m }{
    #1 {,} #2
}
\NewDocumentCommand{\interval}{ s m >{\SplitArgument{1}{,}}m m o }{
    \IfBooleanTF{#1}{
        \left#2 \INTERVALINNARDS #3 \right#4
    }{
        \IfValueTF{#5}{
            #5{#2} \INTERVALINNARDS #3 #5{#4}
        }{
            #2 \INTERVALINNARDS #3 #4
        }
    }
}

\geometry{
 a4paper,
 total={170mm,257mm},
 left=20mm,
 top=20mm,
 }

\begin{document}

\begin{center}
\Huge{Corrección del tercer examen parcial}\\
\vspace{0.5cm}
\Large{González Cárdenas Ángel Aquilez} \\
\end{center}

\vspace{0.5cm}


\begin{enumerate}
\centering
	\item Usando un método adecuado, calcule las siguientes integrales indefinidas:
	\begin{enumerate}
		\item
		\(\int \ln e^{\frac{1 - sen^2 x}{2}} dx\) 
		\\ \vspace{0.5cm}
		Respuesta: De las propiedades de los logaritmos tenemos que para \(\log_a a^x = x\), entonces
		\begin{align*}
			\int \ln e^{\frac{1 - sen^2 x}{2}} dx = \int \frac{1 - sen^2 x}{2} dx
		\end{align*}
		Y como \( 1 - \sen^2 x = cos^2 x \), reescribiendo
		\begin{align*}
			 = \int \frac{cos^2 x}{2} dx = \frac{1}{2} \int cos^2 x dx
		\end{align*}
		Y por la propiedad \(\cos^2 x = \frac{1}{2} + \frac{1}{2}\cos(2x)\), tenemos
		\begin{align*}
			 = \int [ \frac{1}{2} + \frac{1}{2}\cos(2x) ]dx = \frac{1}{2} \int dx + \frac{1}{2} \int \cos(2x) dx
		\end{align*}
		De modo que
		\begin{align*}
			\int \ln e^{\frac{1 - sen^2 x}{2}} dx = \frac{1}{2}[\frac{1}{2} x + \frac{1}{4} sen (2x)] + C =\frac{1}{4} x + \frac{1}{8} sen (2x) + C
		\end{align*}

		\item
		\(\int \dfrac{\arctan \frac{t}{2}}{4 + t^2} dt\)
		\\ \vspace{0.5cm}
		Respuesta: Haciendo 
		\begin{align*}
			u= \arctan \frac{t}{2} \implies du=\frac{\frac{dt}{2}}{1 + \frac{t^2}{4}} = \frac{dt}{2(1+ \frac{t^2}{4})} = \frac{dt}{2 + \frac{1}{2} t^2}
		\end{align*}
		Multiplicando ambos lados de la igualdad por $\frac{1}{2}$ tenemos
		\[ \frac{du}{2} = \frac{dt}{4 + t^2}\]
		Por lo que
		\begin{align*}
			\int \frac{\arctan \frac{t}{2}}{4 + t^2} dt = \frac{1}{2} \int udu = \frac{u^2}{4} + C
		\end{align*}
		Y regresando la sustitución tenemos que el resultado nos da
		\begin{align*}
			 = \frac{1}{4} \arctan \frac{t}{2} + C
		\end{align*}

	\end{enumerate}
	\newpage
	\item Obtenga las siguientes integrales indefinidas según el método más conveniente:
	\begin{enumerate}
		\item \(\int \dfrac{\cos x}{ \sen x + \sen^3 x} dx \)
		\\ \vspace{0.5cm}
		Respuesta: Haciendo \(u = \sen x \implies du=\cos x dx\), tenemos
		\begin{align*}
			\int \frac{\cos x}{ \sen x + \sen^3 x} dx = \int \frac{1}{u + u^3} du
		\end{align*}
		Utilizando el método de funciones racionales, descomponemos la expresión como
		\begin{align*}
			\frac{1}{u + u^3} = \frac{A}{u} + \frac{Bu + C}{u^2 + 1}
		\end{align*}
		De donde \\
		\[ 1 = A(u^2 + 1) + (Bu + C)u = Au^2 + Bu^2 + Cu + A \]
		Para $u^2 = 0 \implies A+B =0$ \\
		Para $u=0 \implies C = 0$,\\
		y para los términos constantes tenemos $1 = A$, de modo que 
		\[ A+B=0 \implies A =-B = -1\]\\
		Por lo tanto, reescribiendo el cambio de variable tenemos que
		\begin{align*}
			\int \frac{1}{u + u^3} du = \int \frac{du}{u} - \int \frac{udu}{u^2 +1}
		\end{align*}
		Para el segundo miembro tenemos que 
		\begin{align*}
			v = u^2 + 1 \implies dv=2udu, \frac{1}{2}dv = udu
		\end{align*}
		Por lo tanto
		\begin{align*}
			= \int \frac{du}{u} - \frac{1}{2} \int \frac{dv}{v} = \ln \mid u \mid - \frac{1}{2} \ln \mid v \mid + C
		\end{align*}
		De modo que 
		\begin{align*}
			 = \ln \mid u \mid - \ln \mid u^2 + 1 \mid +C = \ln \mid \sen x \mid - \frac{1}{2}  \ln \mid \sen^2 x +1 \mid +C
		\end{align*}
		
		\item $\int \dfrac{e^{-x}}{(e^{-2x} + 1)^{\frac{3}{2}}} dx$
		\\ \vspace{0.5cm}
		Respuesta: Como
		\begin{align*}
			\int \frac{e^{-x}}{(e^{-2x} + 1)^{\frac{3}{2}}} dx = \int \frac{dx}{e^x(e^{-2x} + 1)^\frac{3}{2}}
		\end{align*}
		hacemos $u = e^x \implies du = e^x dx, \frac{du}{u^2} = \frac{dx}{e^x}$, de modo que
		\begin{align*}
			=\int \frac{1}{(\frac{1}{u^2} + 1)^\frac{3}{2} u^2} du
		\end{align*}
		Luego, haciendo $v = \frac{1}{u} \implies dv= -\frac{1}{u^2} du$, y utilizando el método de sustitución trigonométrica hacemos
		\[	v = \tan \theta \implies dv = \sec^2 \theta d\theta \] \\
		Y de la propiedad trigonométrica
		\[\sec^2 x - \tan^2 x = 1 \implies \sec^2 x = 1+\tan^2 x\]\\

		De modo que 

		\begin{align*}
			= -\int \frac{1}{(v^2 + 1)^\frac{3}{2}}dv = -\int \frac{1}{(\tan^2 \theta +1)^\frac{3}{2}} \sec^2 \theta d\theta = -\int \frac{\sec^2 \theta d\theta}{(\sec^2 \theta)\frac{3}{2}} = -\int \frac{\sec^2 \theta}{\sec^3 \theta} d\theta
		\end{align*}

		Lo que resulta en 

		\begin{align*}
			=-\int \frac{d\theta}{\sec \theta} = -\int \frac{d\theta}{\frac{1}{\cos \theta}} = -\int \cos \theta d\theta = -\sen \theta +C
		\end{align*}

		Y de la sustitución trigonométrica tenemos que $\theta = \arctan v$, y $\tan \theta = \frac{v}{1}$, así 
		\[\sen \theta = \frac{v}{\sqrt{v^2 +1}}\]
		\\ De modo que
		\begin{align*}
			= -\sen \theta +C = -\frac{v}{\sqrt{v^2 +1}} = - \frac{\frac{1}{u}}{\frac{\sqrt{\frac{1}{u^2} + 1}}{1}} +C = -\frac{1}{u\sqrt{\frac{1}{u^2} + 1}}
		\end{align*}
		Por lo que finalmente
		\[= -\frac{e^{-x}}{\sqrt{e^{-x} + 1}} + C\]
		
	\end{enumerate}
	\item Resuelve las siguientes integrales definidas según el método más conveniente:
	\begin{enumerate}
		\item $\int_{\frac{2\pi}{3}}^{\pi} \dfrac{\sen \frac{1}{2} t}{1 + \cos\frac{1}{2} t} dt$
		\\ \vspace{0.5cm}
		Respuesta: De la integral indefinida $\int \dfrac{\sen \frac{1}{2} t}{1 + \cos\frac{1}{2} t} dt$, hacemos $u = \frac{1}{2} t \implies du=\frac{1}{2}dt$, $2du = dt$, de modo que
		\begin{align*}
			= 2\int \frac{\sen u}{1 + \cos u} du
		\end{align*}
		Luego, haciendo $v = \cos u + 1 \implies dv= -\sen udu$, de modo que
		\begin{align*}
			=-2\int \frac{dv}{v} = -2\ln \mid v \mid +C = -2 \ln \mid \cos u + 1 \mid +C = -2 \ln \mid \cos \frac{1}{2}t + 1 \mid + C
		\end{align*}
		Como $\cos \frac{\pi}{2} = 0$ y $\cos \frac{\pi}{3} = \frac{1}{2}$, para la integral original tenemos que 
		\begin{align*}
			\int_{\pi}^{\frac{2\pi}{3}} \frac{\sen \frac{1}{2} t}{1 + \cos\frac{1}{2} t} dt = [-2\ln \mid \cos \frac{\pi}{2} + 1\mid] + [2\ln \mid \cos \frac{\pi}{2} + 1\mid] \approx 0.81
		\end{align*}
		\item $\int_{0}^{16} \sqrt{4 - \sqrt{x}} dx$
		\\ \vspace{0.5cm}
		Respuesta: De la integral indefinida $\int \sqrt{4 - \sqrt{x}} dx$, haciendo $u = \sqrt{x} \implies du = \frac{1}{2\sqrt{x}}dx$, $udu = \frac{dx}{2}$, de modo que

		\begin{align*}
			= 2\int \sqrt{4 - u}udu
		\end{align*}
		Luego, haciendo $v = 4 -u \implies dv = -du$, por lo cual
		\begin{align*}
			=2\int (v-4) \sqrt{v} dv = 2\int [v^{\frac{3}{2}} - 4\sqrt{v}]dv = 2\int v^{\frac{3}{2}}dv - 8\int \sqrt{v}dv = \frac{4}{5}v^{\frac{5}{2}} - \frac{16}{5}v^{\frac{3}{2}} + C
		\end{align*}
		Por lo tanto, tenemos que
		\begin{align*}
			=\frac{4}{5}(4 - \sqrt{x})^\frac{5}{2} -\frac{16}{3}(4-\sqrt{x})^\frac{3}{2} +C
		\end{align*}
		Finalmente, de la integral original tenemos el resultado 
		\begin{align*}
			\int_{0}^{16} \sqrt{4 - \sqrt{x}} dx &= [\frac{4}{5}(4 - \sqrt{16})^\frac{5}{2} -\frac{16}{3}(4-\sqrt{16})^\frac{3}{2}] - [\frac{4}{5}(4 - \sqrt{0})^\frac{5}{2} -\frac{16}{3}(4-\sqrt{0})^\frac{3}{2}] \\& \approx -17.06
		\end{align*}

	\end{enumerate}

\end{enumerate}


\end{document}




      
    


    
    
        
        