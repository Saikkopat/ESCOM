\documentclass[a4paper,12pt]{report}
\usepackage{geometry}
\usepackage{graphicx}
\usepackage{tikz}
\usepackage{float}
\usepackage[document]{ragged2e}
\usepackage[utf8]{inputenc}
\usepackage[T1]{fontenc}
\usepackage[spanish]{babel}
\renewcommand{\shorthandsspanish}{}
\usepackage{amsmath}
\usepackage{pgfplots}
\usepackage{xparse}
\usepackage{dsfont}

\graphicspath{ {/home/saikkopat/Documents/school/CAL/EX1/} }

\NewDocumentCommand{\INTERVALINNARDS}{ m m }{
    #1 {,} #2
}
\NewDocumentCommand{\interval}{ s m >{\SplitArgument{1}{,}}m m o }{
    \IfBooleanTF{#1}{
        \left#2 \INTERVALINNARDS #3 \right#4
    }{
        \IfValueTF{#5}{
            #5{#2} \INTERVALINNARDS #3 #5{#4}
        }{
            #2 \INTERVALINNARDS #3 #4
        }
    }
}

\geometry{
 a4paper,
 total={170mm,257mm},
 left=20mm,
 top=20mm,
 }

\begin{document}

\begin{center}
\Huge{Corrección del primer examen de Cálculo}\\
\vspace{0.5cm}
\Large{González Cárdenas Ángel Aquilez} \\
\end{center}

\vspace{0.5cm}

\begin{enumerate}

\item ¿Qué valor de $x$ corresponde la solución de la inecuación $\mid 2x + 5 \mid \leq \mid x-3 \mid$? \\

Por la caracterización del valor absoluto tenemos que

\[
	\sqrt{(2x+5)^2} \leq \sqrt{(x-3)^2}
\]
Luego, elevando al cuadrado ambos miembros de la desigualdad

\[
	(2x+5)^2 \leq (x-3)^2
\]

Desarrollando los binomios tenemos

\vspace{-0.5cm}

\begin{align*}
	4x^2 +20x +25 \leq x^2 -6x +9 \\
	3x^2 +26x +16 \leq 0 \\
	3x^2 +24x +2x +16 \leq 0 \\
	3x(x + 8) + 2(x + 8) \leq 0 \\
	(3x + 2)(x + 8) \leq 0
\end{align*}

De donde, obteniendo los puntos críticos tenemos

\begin{align*}
	3x+2 = 0 \implies x = -\frac{2}{3} \\
	x+8 = 0 \implies x = -8
\end{align*}

Luego, con los intervalos

\[  \interval({-\infty,-8}] \text{,} \interval[{-8, -\frac{2}{3}}] \text{,} \interval[{-\frac{2}{3},\infty}) \]


\bgroup
\def\arraystretch{1.5}%
\begin{table}[ht!]
\begin{center}
\begin{tabular}{ c | c  c  c }
		& $\interval({-\infty,-8}]$ & $\interval[{-8, -\frac{2}{3}}]$ & $\interval[{-\frac{2}{3},\infty})$ \\ [0.5ex]
	& $x=-9$ & $x=-4$ 	& $x=0$ \\ 	\hline
	$x+8$ & - & + & + \\
	$3x+2$ & - & - & + \\
	$(x+8)(3x+2)\leq 0$ & + & - & + \\ 
\end{tabular}
\end{center}
\end{table}


Por lo tanto, la solución para la desigualdad $\mid 2x + 5 \mid \leq \mid x-3 \mid$ es el intervalo \\

\[ \interval[{-8,-\frac{2}{3}}] = \{ x \in \mathds{R} \mid -8 \leq x \leq -\frac{2}{3} \} \]


\item Con $a,b \in \mathds{R}$, demuestre que si $0<a<b$ entonces $a<\sqrt{ab}<\dfrac{a+b}{2}<b$.\\

\vspace{0.5cm}

De $0<a<b \implies a<b$ con $a>0$, luego

\[ a^2 < ab \implies \sqrt{a^2} < \sqrt{ab}\]
del cual

\begin{equation}
	a < \sqrt{ab}
\end{equation}

Por otra parte, para $\sqrt{ab}<\dfrac{a+b}{2}$, partiendo de

\begin{align*}
0<a<b \implies & a<b 
\\ & 0<b-a 
\\ & 0<(b-a)^2 
\\ & 0<b^2 -2ab + a^2
\\ & 2ab< b^2 + a^2
\\ & 4ab< b^2 + a^2 + 2ab
\\ & 4ab < (b+a)^2
\\ & \sqrt{4ab} < \sqrt{(b+a)^2}
\end{align*}
Como $a>0$ y $b>0$
\[2\sqrt{ab} < a+b\]
\begin{equation}
\sqrt{ab} < \frac{a+b}{2}
\end{equation}

Luego, para $\frac{a+b}{2} <b$, de $0<a<b$
\[a<b \implies a+b < 2b\]
\begin{equation}
 \frac{a+b}{2} < b
\end{equation}


Finalmente, por la \emph{propiedad transitiva}, de (1), (2), (3) tenemos:

\[ a<\sqrt{ab}<\frac{a+b}{2}<b \]

\newpage

\item De $f(x) = \mid x^2 - 1 \mid; g(x) = \mid x - 1\mid$, obtenga la función resultante y dominio en $(f-g)(x)$, $(f \cdot g)(x)$, $(\dfrac{f}{g})(x)$ y $(f \circ g)(x)$.

\vspace{0.5cm}

De

\[
f(x) = \mid x^2 -1 \mid = 
	\begin{cases}	
		x^2-1 & \text{ si } x \leq -1 \text{ o } x\geq 1 \\
		1-x^2 & \text{ si } -1 < x < 1
	\end{cases}
\]

y

\[
g(x) = \mid x -1 \mid = 
	\begin{cases}	
		x-1 & \text{ si } x \geq 1 \\
		1-x & \text{ si } x < 1
	\end{cases}
\]

Para

\[
(f-g)(x) = 
	\begin{cases}	
		x^2 +x -2 & \text{ si } x \leq -1 \\
		x - x^2 & \text{ si } -1 < x < 1 \\
		x^2 - x & \text{ si } x \geq 1
	\end{cases}
\]
con \emph{D}$\{ (f-g)(x) \} = \mathds{R} $.\\

Luego, para

\[
(f \cdot g)(x) = 
	\begin{cases}	
		-x^3 + x^2 +x -1 & \text{ si } x \leq -1 \\
		x^3 - x^2 - x + 1 & \text{ si } x > -1
	\end{cases}
\]
con \emph{D}$\{ (f \cdot g)(x) \} = \mathds{R} $.\\

Por otra parte, para $(\dfrac{f}{g})(x) \implies g(x) \neq 0 \implies x \neq 1$, así

\[
(\dfrac{f}{g})(x) = 
	\begin{cases}	
		-x -1 & \text{ si } x \leq -1 \\
		x + 1 & \text{ si } x > -1
	\end{cases}
\]

con \emph{D}$\{ \dfrac{f}{g})(x) \} = \mathds{R} $.\\

Finalmente, para 
\begin{align*}
(f \circ g)(x)
& = f(g(x))
\\ & = f(\sqrt{(x-1)^2} -1)
\\ &  = \mid (\sqrt{(x-1)^2})^2 -1\mid
\end{align*}

que por la caracterización del valor absoluto se tiene que\\

\[ \mid (x-1)^2 -1\mid = \mid x^2 -2x \mid\]

con \emph{D}$\{ (f \circ g)(x) \} = \mathds{R} $.\\


\newpage


\item Determine de forma analítica toda la información de $f(x) = \dfrac{x}{\sqrt{x^2 -9}}$ que incorpore dominio , simetría, $f^-1(x)$, punto de intersección en los ejes y asíntotas si existen. Confirme sus conclusión y bosqueje la función en una gráfica.\\
\vspace{0.5cm}

Como $\sqrt{x^2 - 9} > 0 \implies x^2 -9 >0 \implies \mid x \mid >3$
de donde 
\[\mid x \mid >3 \iff x<-3 \text{ o } x>3\]
con \emph{D}$\{ (f(x) \} = \interval({-\infty,-3}) \cup \interval({3,\infty}) $.\\
\vspace{0.5cm}
Con $x = -3$ y $x=3$ se tienen \emph{asíntotas verticales}.
\vspace{0.5cm}

Para 	

\[
	\lim_{x \to -3^-} f(x) = \lim_{x \to -3^-} \frac{x}{\sqrt{x^2 -9}} = \frac{-3}{0^+} = -\infty
\]

y

\[
	\lim_{x \to 3^+} f(x) = \lim_{x \to 3^-} \frac{x}{\sqrt{x^2 -9}} = \frac{3}{0^+} = \infty
\]
Para la simetría, con $x=-x \implies $\\

\[
	f(-x) = \frac{-x}{\sqrt{(-x^2) -9}} = -\frac{x}{\sqrt{x^2 -9}} = -f(x)
\]

Por lo tanto, $f(x)$ es \emph{impar}.\\

Por otro lado, para $f^-1(x)$, haciendo

\[x = f^-1 (y) \iff y = f(x)\]

de donde 

\begin{align*}
y = \frac{x}{\sqrt{x^2 -9}} \implies & y^2 = \frac{x^2}{x^2 -9}
\\ & y^2 (x^2 -9) = x^2
\\ & y^2 x^2 - x^2 = 9y^2
\\ & x^2 (y^2 +1)  = 9y^2
\\ & x^2 = \frac{9y^2}{y^2 -1}\\
\implies  f^-1 (y) = x = \frac{3y}{\sqrt{y^2 -1}} \\
 f^-1(x) = \frac{3x}{\sqrt{x^2 -1}}
\end{align*}

Luego, para $x=0 \implies y = \dfrac{0}{\sqrt{0 - 9}} \notin \mathds{R}$, $f(x)$ no intersecta al eje \emph{Y}.\\

Así mismo, haciendo $0 = \dfrac{x}{\sqrt{x^2 -9}} \notin \mathds{R}$, $f(x)$ no intersecta al eje \emph{X}.\\

Luego, para las asíntotas horizontales tenemos que\\

\begin{align*}
\lim_{x \to +\infty} f(x) = & \lim_{x \to +\infty} \frac{x}{\sqrt{x^2 -9}} = \lim_{x \to +\infty} \frac{x}{\sqrt{x^2(1 -\frac{9}{x^2})}}
\\ &  \lim_{x \to +\infty} \frac{x}{\mid x \mid \sqrt{1 - \frac{9}{x^2}}} = \lim_{x \to +\infty} \frac{1}{\sqrt{1 - \frac{9}{x^2}}}
\\ & \frac{1}{1} = 1
\end{align*}

y

\[
	\lim_{x \to -\infty} f(x) = \lim_{x \to -\infty} \frac{-1}{\sqrt{1 - \frac{9}{x^2}}} = -1
\]

Por lo tanto, $f(x)$ tiene dos asíntotas horizontales en $y = 1$ y $y = -1$.

De la siguiente gráfica se comprueba lo anterior:\\
\vspace{1cm}

\pgfplotsset{every axis/.append style={
font=\small,
thin,
tick style={ultra thin}}}
\begin{tikzpicture}
\centering
\begin{axis}[
		width=1\textwidth,
    height=1\textwidth,
    axis lines = center,
    xlabel = $x$,
    ylabel = {$y$},
    xmax = {10},
    xmin = {-10},
    ymax = {10},
    ymin = {-10},
    xtick={-8,-6,-4,-2, 2,4,6,8},
    ytick={-8, -6, -4, -2, 2, 4, 6, 8}
]
\addplot [
    domain=-10:-3,
    samples=3000,
    color=red, smooth
]
{(x)/(sqrt(x^2 - 9))};

\addplot [
    domain=3:10,
    samples=3000,
    color=red,
]
{(x)/(sqrt(x^2 - 9))};

\addplot[black,dashed,domain = -10:10,samples = 2] {-1};
\addplot[black,dashed,domain = -10:10,samples = 2] {1};

% Vertical asymptotes at -3 and 3
\draw[dashed] ({axis cs:-3,0}|-{rel axis cs:0,0}) -- ({axis cs:-3,0}|-{rel axis cs:0,1});
\draw[dashed] ({axis cs:3,0}|-{rel axis cs:0,0}) -- ({axis cs:3,0}|-{rel axis cs:0,1});
% Horizontal asymptotes at -1 and 1
%
\end{axis}
\end{tikzpicture}


\newpage

\item Calcule el valor de $\lim\limits_{x \to -3} \sqrt{\dfrac{y^2 -9}{2y^2 +7y +3}}$ usando álgebra preliminar y técnicas de limites.\\
\vspace{0.5cm}

Factorizando tenemos que

\begin{align*}
\lim_{x \to -3} \sqrt{\frac{y^2 -9}{2y^2 +7y +3}} = & \lim_{x \to -3} \sqrt{\frac{(y+3)(y-3)}{(y+3)(2y+1)}} 
\\ & = \lim_{x \to -3} \sqrt{\frac{y-3}{2y+1}}
\\ & = \sqrt{\frac{6}{5}} = \sqrt{\frac{30}{5}}
\end{align*}

\item Calcule el valor de $\lim\limits_{x \to 0} (\dfrac{\csc 2x}{\cot 2x} + \dfrac{\sin 3x}{3x^2 +2x}) $ usando álgebra preliminar y técnicas de limites.\\

\vspace{0.5cm}

Haciendo

\begin{align*}
\lim_{x \to 0} (\frac{\csc 2x}{\cot 2x} + \frac{\sin 2x}{3x^2 +2x}) & = \lim_{x \to 0} \frac{\csc 2x}{\cot 2x} + \lim_{x \to 0} \frac{\sin 3x}{3x^2 +2x}
\\ & = \lim_{x \to 0} \dfrac{\dfrac{1}{\sin 2x}}{\dfrac{\cos 2x}{\sin 2x}} + \lim_{x \to 0} \frac{\sin 3x}{3x(x+ \frac{2}{3})}
\end{align*}

Como \[\lim_{x \to 0} \dfrac{\sin x}{x} = 1\]

\[\lim_{x \to 0} \frac{1}{\cos x} + \lim_{x \to 0} \frac{1}{x+ \frac{2}{3}}\]

Luego, evaluando el limite 

\[
	1 + \frac{3}{2} = \frac{5}{2}
\]



\end{enumerate}





\end{document}