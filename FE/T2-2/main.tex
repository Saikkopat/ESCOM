\documentclass[a4paper,12pt]{article}
\usepackage[a4paper, margin=2.5cm]{geometry}
\usepackage[pdftex]{graphicx}
\usepackage{tikz}
\usepackage{pgfplots}
\usepackage{enumitem}
\usepackage{float}
\usepackage[document]{ragged2e}
\usepackage[utf8]{inputenc}
\usepackage[T1]{fontenc}
\usepackage[spanish,es-tabla]{babel}
\renewcommand{\shorthandsspanish}{}
\usepackage{xurl}
\usepackage{lipsum}
\usepackage{mwe}
\usepackage{multicol}
\usepackage{siunitx}
\usepackage{listings}
\usepackage{enumitem}
\usepackage{amsmath}
\usepackage{listings}
\usepackage{tabularray}

\begin{document}

\begin{titlepage}
	\begin{tikzpicture}[overlay, remember picture]
		\path (current page.north east) ++(-0.3,-1.5) node[below left] {\includegraphics[width=0.35\textwidth]{/home/saikkopat/Documents/LOGOS IPN/EscudoESCOM}};
	\end{tikzpicture}
	\begin{tikzpicture}[overlay, remember picture]
		\path (current page.north west) ++(1.5,-1) node[below right] {\includegraphics[width=0.2\textwidth]{/home/saikkopat/Documents/LOGOS IPN/logo}};
	\end{tikzpicture}
	\begin{center}
		\vspace{-1.5cm}
		{\LARGE Instituto Politécnico Nacional\par}
		\vspace{.5cm}
		{\LARGE Escuela Superior de Cómputo\par}
		\vspace{2.5cm}
		{\large Unidad de aprendizaje:}\\{\Large Fundamentos económicos\par}
		\vspace{7cm}
		{\scshape\Huge Actividad 2:\par}
		{\itshape\Large Producción con un factor variable\par}
		\vfill
		{\Large Alumno: González Cárdenas Ángel Aquilez\par}
		\vspace{1cm}
		{\Large Boleta: 2016630152\par}
		\vspace{1cm}
		{\Large Grupo: 2CV2\par}
		\vspace{1cm}
		{\Large Profesora: Villegas Navarrete Sonia\par}
		\vfill
	\end{center}
\end{titlepage} 

\newpage

\textbf{\Large Ejercicio:}\\
\vspace{0.5cm}

Cuando la tierra es fija, pero el trabajo es variable, la empresa solo puede producir más incrementado su cantidad de trabajo.\par

\vspace{0.5cm}

De la Tabla 1, las tres primeras columnas muestras una función hipotética de producción a corto plazo
para trigo. Tenemos que decidir ¿cuánto trabajo vamos a contratar y cuánto frijol vamos a
producir? \\

\vspace{0.5cm}

Con base en la Tabla 1: \\

\begin{enumerate}[label=\arabic*)]
	\item Determinar el \emph{PMe} y el \emph{PMa} del trabajo.
	\item Trace las curvas del \emph{PT}, \emph{PMe} y \emph{PMa} de trabajo.
	\item Interprete las curvas considerando sus tres etapas.
\end{enumerate}


\bgroup
\def\arraystretch{1.5}%
\begin{table}[ht!]
\setlength\tabcolsep{3pt}
\begin{center}
\begin{tblr}{|c c c c c|}
\hline
		Tierra & Trabajo & Producto Total & Producto Medio & Producto Marginal\\
		(T)&(L)&(PT)&(PMe)&(PMa) \\ \hline
		2 & 0 & 0 & - & - \\ 
		2 & 1 & 4 & 4 &  4 \\ 
		2 & 2 & 10 & 5 & 6 \\ 
		2 & 3 & 18 & 6 & 8 \\ 
		2 & 4 & 24 & 6 & 6 \\ 
		2 & 5 & 28 & 5.6 & 4 \\ 
		2 & 6 & 30 & 5 & 2 \\ 
		2 & 7 & 30 & 4.2 & 0 \\ 
		2 & 8 & 28 & 3.5 & -2 \\ 
		2 & 9 & 24 & 2.6 & -4 \\
\hline
\end{tblr}
\label{table:1}
\caption{Función hipotética de producción a corto plazo para trigo}
\end{center}
\end{table}

\begin{enumerate}

\item
De la Tabla 1 se obtiene el \emph{Producto Medio (PMe)} de la division entre el \emph{Producto Total (PT)} y la cantidad de \emph{Trabajo (L)}:

\[
	PMe = \frac{PT}{L}
\]

Y al aplicar la fórmula sobre los valores de la Tabla 1 tenemos que:

\begin{align*}
	PMe_1 &= \frac{4}{1} = 4 \\
	PMe_2 &= \frac{10}{2} = 5 \\
	PMe_3 &= \frac{18}{3} = 6 \\
	PMe_4 &= \frac{24}{4} = 6 \\
	PMe_5 &= \frac{28}{5} = 5.6 \\
	PMe_6 &= \frac{30}{6} = 5 \\
	PMe_7 &= \frac{30}{7} = 4.2 \\
	PMe_8 &= \frac{28}{8} = 3.5 \\
	PMe_9 &= \frac{24}{9} = 2.6 \\
\end{align*}

Como no es posible dividir la cantidad de producto total cuando \emph{(L)} cuando es igual a 0, se omite algún resultado por la indeterminación. \\
Luego, para obtener el \emph{Producto Marginal (PMa)} de la fórmula: 

\[
	PMa = \frac{\Delta PT}{\Delta L} = \frac{PT_{L2} - PT_{L1}}{L_2-L_1}
\]

De donde: \\

\begin{align*}
	PMa_{0-1} &= \frac{4 - 0}{1 - 0} = 4 \\
	PMa_{1-2} &= \frac{10 - 4}{2 - 1} = 6 \\
	PMa_{2-3} &= \frac{18 - 10}{3 - 2} = 8 \\
	PMa_{3-4} &= \frac{24 - 18}{4 - 3} = 6 \\
	PMa_{4-5} &= \frac{28 - 24}{5 - 4} = 4 \\
	PMa_{5-6} &= \frac{30 - 28}{6 - 5} = 2 \\
	PMa_{6-7} &= \frac{30 - 30}{7 - 6} = 0 \\
	PMa_{7-8} &= \frac{28 - 30}{8 - 7} = -2 \\
	PMa_{8-9} &= \frac{24 - 28}{9 - 8} = -4 \\
\end{align*}


\item
En la Figura 1 se aprecian los datos obtenidos de forma gráfica: 

\begin{figure}[h!]
	\centering
	
\begin{tikzpicture}

\begin{axis} [
	legend style={legend pos=outer north east},
	width=12cm,	
	grid=both,
	ymin=-5,
	xmin=0, xmax=10,
	xtick={1,...,10},
	title=Producción con un factor variable,
	ylabel=Producto Total (PT) Producto Medio (PMe) y Producto Marginal (PMa),
	xlabel=Trabajo (L),
]

\addplot [color=orange] coordinates {
			(0,0)
			(1,4)
			(2,10)
			(3,18)
			(4,24)
			(5,28)
			(6,30)
			(7,30)
			(8,28)
			(9,24)

};
\addlegendentry{PT}

\addplot[color=red]
coordinates {
	(1,4)
	(2,5)
	(3,6)
	(4,6)
	(5,5.6)
	(6,5)
	(7,4.2)
	(8,3.5)
	(9,2.6)
};
\addlegendentry{PMe}

\addplot [color=green] coordinates {
			(.5,4)
			(1.5,6)
			(2.5,8)
			(3.5,6)
			(4.5,4)
			(5.5,2)
			(6.5,0)
			(7.5,-2)
			(8.5,-4)

};
\addlegendentry{PMa}

\draw[dashed] ({axis cs:3.5,0}|-{rel axis cs:0,0}) -- ({axis cs:3.5,0}|-{rel axis cs:0,1});
\draw[dashed] ({axis cs:6.5,0}|-{rel axis cs:0,0}) -- ({axis cs:6.5,0}|-{rel axis cs:0,1});


\end{axis}
\end{tikzpicture}
\caption{Etapas de la producción}
\end{figure}

\item Se concluye de la Tabla 1 y la Figura 1 que se alcanza el máximo en la producción con un trabajo de 5 a 7 y una producción de 28 a 30.

\end{enumerate}

\vspace{5cm}
A continuación se anexa el ejercicio realizado en clase:

\end{document}