\documentclass[a4paper,12pt]{report}
\usepackage{graphicx}
\usepackage{tikz}
\usepackage{float}
\usepackage[document]{ragged2e}
\usepackage[utf8]{inputenc}
\usepackage[T1]{fontenc}
\usepackage[spanish]{babel}
\renewcommand{\shorthandsspanish}{}
\usepackage{xurl}

\usepackage{listings}

\begin{document}

\begin{titlepage}
	\begin{tikzpicture}[overlay, remember picture]
		\path (current page.north east) ++(-0.3,-1.5) node[below left] {\includegraphics[width=0.4\textwidth]{/home/saikkopat/Documents/LOGOS IPN/EscudoESCOM}};
	\end{tikzpicture}
	\begin{tikzpicture}[overlay, remember picture]
		\path (current page.north west) ++(1.5,-1) node[below right] {\includegraphics[width=0.23\textwidth]{/home/saikkopat/Documents/LOGOS IPN/logo}};
	\end{tikzpicture}
	\begin{center}
		\vspace{-3cm}
		{\LARGE Instituto Politécnico Nacional\par}
		{\Large Escuela Superior de Cómputo\par}
		\vspace{7cm}
		{\scshape\Huge Resumen\par}
		{\itshape\Large Definición de economía\par}
		\vfill
		{\Large Alumno: González Cárdenas Ángel Aquilez\par}
		\vspace{1cm}
		{\Large Boleta: 2016630152\par}
		\vspace{1cm}
		{\Large Grupo: 2CV2\par}
		\vspace{1cm}
		{\Large Unidad de aprendizaje: Fundamentos económicos\par}
		\vspace{1cm}
		{\Large Profesora: Villegas Navarrete Sonia\par}
		\vfill
	\end{center}
\end{titlepage} 

\newpage

\textbf{\Large Definición de economía}\\
\vspace{0.5cm}

La incapacidad de los seres vivos para conseguir todo lo que desean se denomina \emph{escasez}. Lo que es asequible para los seres humanos está limitado por los ingresos y los precios que deben pagar, así como el tiempo está limitado a 24 horas por día. A su vez, lo que es asequible para los gobiernos está limitado por los impuestos que recolecta. Así, lo que la sociedad puede obtener se encuenta limitado por los \emph{recursos productivos} disponibles.\\
En virtud de lo que no se puede obtener, se llega a una \emph{elección}.\\
Las elecciones se concilian con los \emph{incentivos} para estimular una acción o un castigo. Por ejemplo, los precios de los productos funcionan como incentivos, llegando a determinar la cantidad de productos a producir y vender si su precio incentiva a las personas a comprarlo de manera más fácil.

\newpage

\textbf{Bibliografía}

\begin{enumerate}
	\item Parkin, M. \& Loria, E. (2015) Microeconomía versión para Latinoamérica (544 pág). México, Pearson Educación.
\end{enumerate}

\end{document}