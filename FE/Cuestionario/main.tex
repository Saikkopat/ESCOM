\documentclass[a4paper,12pt]{report}
\usepackage{graphicx}
\usepackage{tikz}
\usepackage{float}
\usepackage[document]{ragged2e}
\usepackage[utf8]{inputenc}
\usepackage[T1]{fontenc}
\usepackage[spanish]{babel}
\renewcommand{\shorthandsspanish}{}
\usepackage{xurl}

\usepackage{listings}

\begin{document}

\begin{titlepage}
	\begin{tikzpicture}[overlay, remember picture]
		\path (current page.north east) ++(-0.3,-1.5) node[below left] {\includegraphics[width=0.4\textwidth]{/home/saikkopat/Documents/LOGOS IPN/EscudoESCOM}};
	\end{tikzpicture}
	\begin{tikzpicture}[overlay, remember picture]
		\path (current page.north west) ++(1.5,-1) node[below right] {\includegraphics[width=0.23\textwidth]{/home/saikkopat/Documents/LOGOS IPN/logo}};
	\end{tikzpicture}
	\begin{center}
		\vspace{-3cm}
		{\LARGE Instituto Politécnico Nacional\par}
		{\Large Escuela Superior de Cómputo\par}
		\vspace{2.5cm}
		{\large Unidad de aprendizaje:}\\{\Large Fundamentos económicos\par}
		\vspace{3cm}
		{\scshape\Huge Cuestionario:\par}
		{\itshape\Large Política económica\par}
		\vfill
		\vspace{1cm}
		{\itshape\Large Equipo 3\par}
		\vspace{.5cm}
		{\Large Integrantes: \\González Cárdenas Ángel Aquilez\par}
		\vspace{.5cm}
		{\Large Boleta: 2016630152\par}
		\vspace{.5cm}
		{\Large Hernández Díaz Roberto Ángel\par}
		\vspace{.5cm}
		{\Large Boleta: 2023630031\par}
		\vspace{1cm}
		{\Large Grupo: 2CV2\par}
		\vspace{1cm}
		{\Large Profesora: Villegas Navarrete Sonia\par}
		\vfill
	\end{center}
\end{titlepage} 

\newpage

\textbf{\Large Cuestionario:}\\

\begin{enumerate}
	\item ¿Cuál es la moneda de uso en la unión europea? \\ Respuesta: El euro
	\item ¿Qué es el cambio nominal? \\ Respuesta: El valor atribuido
	\item ¿Qué régimen monetario tiene México? \\ Respuesta: Flexible
	\item ¿Qué moneda se utiliza en Ecuador de manera oficial? \\ Respuesta: Dolár
	\item ¿Cuándo ocurrió el jueves negro? \\ Respuesta: 24 de Octubre, 1929
	\item ¿Quienes regulan la actividad relacionada con el control sobre el flujo de dinero? \\ Respuesta: Los bancos centrales
	\item ¿Cómo se la llama a la situación: cuando hay un exceso de flujo monetario en un país sin que la economía haya crecido? \\ Respuesta: Inflación
	\item ¿Qué es la política fiscal? \\ Respuesta: Son las decisiones que toma el Gobierno
para aumentar o disminuir las recaudaciones de impuestos, así como también los gastos públicos,
con el fin de mantener la estabilidad económica
	\item ¿Qué significa \emph{OMA}? \\ Respuesta: Operaciones de Mercado Abierto
	\item ¿En qué se divide la política monetaria? \\ Respuesta: Restrictiva y Expansiva
\end{enumerate}


\end{document}