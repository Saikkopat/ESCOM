\documentclass[a4paper,12pt]{article}
\usepackage[a4paper, margin=2.5cm]{geometry}
\usepackage[pdftex]{graphicx}
\usepackage{tikz}
\usepackage{pgfplots}
\usepackage{enumitem}
\usepackage{float}
\usepackage[document]{ragged2e}
\usepackage[utf8]{inputenc}
\usepackage[T1]{fontenc}
\usepackage[spanish,es-tabla]{babel}
\renewcommand{\shorthandsspanish}{}
\usepackage{xurl}
\usepackage{lipsum}
\usepackage{mwe}
\usepackage{multicol}
\usepackage{siunitx}
\usepackage{listings}
\usepackage{enumitem}
\usepackage{amsmath}
\usepackage{listings}
\usepackage{tabularray}
\graphicspath{ {/home/saikkopat/Documents/ESCOM/FE/C10/} }

\begin{document}

\begin{titlepage}
	\begin{tikzpicture}[overlay, remember picture]
		\path (current page.north east) ++(-0.3,-1.5) node[below left] {\includegraphics[width=0.35\textwidth]{/home/saikkopat/Documents/LOGOS IPN/EscudoESCOM}};
	\end{tikzpicture}
	\begin{tikzpicture}[overlay, remember picture]
		\path (current page.north west) ++(1.5,-1) node[below right] {\includegraphics[width=0.2\textwidth]{/home/saikkopat/Documents/LOGOS IPN/logo}};
	\end{tikzpicture}
	\begin{center}
		\vspace{-1.5cm}
		{\LARGE Instituto Politécnico Nacional\par}
		\vspace{.5cm}
		{\LARGE Escuela Superior de Cómputo\par}
		\vspace{2.5cm}
		{\large Unidad de aprendizaje:}\\{\Large Fundamentos económicos\par}
		\vspace{7cm}
		{\scshape\Huge Actividad 10:\par}
		{\itshape\Large Economía círcular\par}
		\vfill
		{\Large Alumno: González Cárdenas Ángel Aquilez\par}
		\vspace{1cm}
		{\Large Boleta: 2016630152\par}
		\vspace{1cm}
		{\Large Grupo: 2CV2\par}
		\vspace{1cm}
		{\Large Profesora: Villegas Navarrete Sonia\par}
		\vfill
	\end{center}
\end{titlepage} 

\newpage

\section*{Cuestionario}
\begin{enumerate}



\item ¿Qué es un modelo económico? \\
Respuesta: Una simplificación de la realidad para comprender fenómenos económicos
\item ¿Por qué principio se caracteriza la economía lineal? \\
Respuesta: Tomar, hacer, desechar.
\item ¿Cuál de estos materiales tarda más en degradarse?\\
Respuesta: Vidrio
\item ¿Qué es la economía circular?\\
Respuesta: Un modelo económico orientado a reciclar, reutilizar y regenerar.
\item Es un principio de la economía circular\\
	Respuesta: Promover la efectividad del sistema, preservar y aumentar el capital natural.
\item La obsolescencia programada, ¿va de acuerdo con los principios de la economía circular? \\
Respuesta: Falso.
\item Una característica clave de una economía circular
\\ Respuesta: Mantener el valor de los productos
\item Ejemplo de una empresa que implementa el sistema producto – servicio
\\ Respuesta: Airbnb, Xbox
\item Es un servicio del Cloud Computing
\\ Respuesta: SaaS (Software as a Service)
\item Los servicios SaaS se consideran como un sistema producto – servicio, ¿de qué tipo?
\\ Respuesta: Orientados al uso

\end{enumerate}


\newpage

A continuación se anexa la evidencia del ejercicio realizado en clase:
\vspace{1cm}
\begin{figure}[h!]
\centering
	\includegraphics[width=.5\textwidth]{fig1}
\end{figure}

\end{document}