\documentclass[a4paper,12pt]{report}
\usepackage{graphicx}
\usepackage{tikz}
\usepackage{float}
\usepackage[document]{ragged2e}
\usepackage[utf8]{inputenc}
\usepackage[T1]{fontenc}
\usepackage[spanish]{babel}
\renewcommand{\shorthandsspanish}{}
\usepackage{xurl}

\usepackage{listings}

\begin{document}

\begin{titlepage}
	\begin{tikzpicture}[overlay, remember picture]
		\path (current page.north east) ++(-0.3,-1.5) node[below left] {\includegraphics[width=0.4\textwidth]{/home/saikkopat/Documents/LOGOS IPN/EscudoESCOM}};
	\end{tikzpicture}
	\begin{tikzpicture}[overlay, remember picture]
		\path (current page.north west) ++(1.5,-1) node[below right] {\includegraphics[width=0.23\textwidth]{/home/saikkopat/Documents/LOGOS IPN/logo}};
	\end{tikzpicture}
	\begin{center}
		\vspace{-3cm}
		{\LARGE Instituto Politécnico Nacional\par}
		{\Large Escuela Superior de Cómputo\par}
		\vspace{2.5cm}
		{\large Unidad de aprendizaje:}\\{\Large Fundamentos económicos\par}
		\vspace{5cm}
		{\scshape\Huge Actividad 1:\par}
		{\itshape\Large La empresa y su problema económico\par}
		\vfill
		\vspace{1cm}
		{\Large Integrantes: \\González Cárdenas Ángel Aquilez\par}
		\vspace{.5cm}
		{\Large Boleta: 2016630152\par}
		\vspace{.5cm}
		{\Large Hernández Díaz Roberto Ángel\par}
		\vspace{.5cm}
		{\Large Boleta: 2023630031\par}
		\vspace{1cm}
		{\Large Grupo: 2CV2\par}
		\vspace{1cm}
		{\Large Profesora: Villegas Navarrete Sonia\par}
		\vfill
	\end{center}
\end{titlepage} 

\newpage

\textbf{\Large La empresa y su problema económico}\\

\vspace{1cm}

Las empresas tienen como objetivo maximizar sus beneficios y utilidades a partir de que pueden cubrir necesidades de una determinada sociedad. Y de forma similar a los consumidores, quienes cuentan con una restricción en lo que pueden adquirir por la renta o ingreso del que disponen, las empresas también se enfrentan a la escasez y la restricción del capital del que disponen.\\

\vspace{0.5cm}

Enfrentar la escasez implica costos de oportunidad. De manera análoga a la tasa marginal de sustitución, nos es posible determinar el costo de oportunidad en la producción gracias al concepto de \emph{frontera de posibilidades de producción}, el cual ilustra la relación entre dos bienes a producir con una cantidad establecida de recursos. \\

\vspace{0.5cm}

Según \emph{Parkin y Loria}, la \emph{frontera de posibilidades de producción} también ilustra la escasez, porque \textit{ es imposible alcanzar los puntos que están más allá de la frontera, [...], en contraste, podemos producir en
cualquier punto ubicado dentro de la FPP y en los
que están sobre ella: puntos alcanzables}.\\

\vspace{0.5cm}

Al tener establecida nuestra frontera de posibilidades de producción, tenemos establecida nuestra producción eficiente, esto es, producir ocupando todos los recursos disponibles. Cualquier punto que se encuentre por debajo implicaría no utilizar todos los recursos, y por encima, que son inalcanzables.\\

\vspace{0.5cm}

Como los recursos de los que disponen las empresas son limitados, la eficiencia se vuelve crítica para la producción. Pero, de todas las posibles combinaciones establecidas por la frontera de posibilidades de producción, ¿cuál es la más eficiente? Nuevamente, de manera análoga al análisis del comportamiento del consumidor (en este caso, la utilidad marginal por unidad monetaria), la combinación que ofrece la mayor utilidad por el menor costo será la más eficiente.\\

\vspace{0.5cm}

Ahora bien, como todas las empresas tienen el mismo objetivo, producen de la manera más eficiente posible, ¿por qué algunas tienen ventaja sobre otras?\\

\vspace{0.5cm}

Cuando una empresa produce lo mismo que otra pero a menor costo, se encuentra en una \emph{ventaja comparativa} sobre el resto, y si es más eficiente, tendrá una \emph{ventaja absoluta}. Sin embargo, no necesariamente una implica a la otra, pues ambas dependen de las condiciones y habilidades implicadas en cada empresa sumado a su costo de oportunidad individual.\\

Para que una empresa pueda sacar provecho de la actividad que realiza, debe concentrarse en aquella en la que tiene una ventaja comparativa sobre el resto, es decir, para lo que sea mejor y le convenga más, lo que significa un proceso de \emph{especialización}.\\

\vspace{0.5cm}

Entre dos empresas que producen de forma eficiente el mismo bien, influye el factor tecnológico, pues la producción aprovecha mejor los recursos con el avance de la tecnología.\\

\vspace{0.5cm}

Ahora bien, cuando una empresa aumenta sus capacidades de producción, aumenta su crecimiento económico, lo que resulta de la acumulación de capital y el cambio tecnológico.\\

Nuevamente, de manera análoga a la teoría del consumidor, cuando una empresa produce más de un bien con la mejor tecnología de la que dispone para la mayor eficiencia posible, nos encontramos con lo que según \emph{Varian}:\\

\begin{quote}
Cuando dos bienes se producen de la manera más eficiente posible, la relación marginal de transformación entre ellos indica el número de unidades de uno de los bienes a las que tiene que renunciar la economía para obtener unidades adicionales del otro.
\end{quote}

\newpage

\subsection*{Bibliografía}

\begin{enumerate}

	\item Parkin, M. \& Loria, E. (2015) Microeconomía versión para Latinoamérica (Capitulo 2: El problema económico). $9^a$ Edición. México, Pearson Educación.

	\item Varian, H. R. (2002). Microeconomía intermedia - Un enfoque actual (página 570) $5^a$ Edición. Antoni Bosch Editor.


\end{enumerate}


\end{document}