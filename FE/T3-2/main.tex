\documentclass[a4paper,12pt]{article}
\usepackage[a4paper, margin=2.5cm]{geometry}
\usepackage[pdftex]{graphicx}
\usepackage{tikz}
\usepackage{pgfplots}
\usepackage{enumitem}
\usepackage{float}
\usepackage[document]{ragged2e}
\usepackage[utf8]{inputenc}
\usepackage[T1]{fontenc}
\usepackage[spanish,es-tabla]{babel}
\renewcommand{\shorthandsspanish}{}
\usepackage{xurl}
\usepackage{lipsum}
\usepackage{mwe}
\usepackage{multicol}
\usepackage{siunitx}
\usepackage{listings}
\usepackage{enumitem}
\usepackage{amsmath}
\usepackage{listings}
\usepackage{tabularray}
\graphicspath{ {/home/saikkopat/Documents/ESCOM/FE/T3-2/} }

\begin{document}

\begin{titlepage}
	\begin{tikzpicture}[overlay, remember picture]
		\path (current page.north east) ++(-0.3,-1.5) node[below left] {\includegraphics[width=0.35\textwidth]{/home/saikkopat/Documents/LOGOS IPN/EscudoESCOM}};
	\end{tikzpicture}
	\begin{tikzpicture}[overlay, remember picture]
		\path (current page.north west) ++(1.5,-1) node[below right] {\includegraphics[width=0.2\textwidth]{/home/saikkopat/Documents/LOGOS IPN/logo}};
	\end{tikzpicture}
	\begin{center}
		\vspace{-1.5cm}
		{\LARGE Instituto Politécnico Nacional\par}
		\vspace{.5cm}
		{\LARGE Escuela Superior de Cómputo\par}
		\vspace{2.5cm}
		{\large Unidad de aprendizaje:}\\{\Large Fundamentos económicos\par}
		\vspace{7cm}
		{\scshape\Huge Actividad 8:\par}
		{\itshape\Large Importancia del comercio internacional\par}
		\vfill
		{\Large Alumno: González Cárdenas Ángel Aquilez\par}
		\vspace{1cm}
		{\Large Boleta: 2016630152\par}
		\vspace{1cm}
		{\Large Grupo: 2CV2\par}
		\vspace{1cm}
		{\Large Profesora: Villegas Navarrete Sonia\par}
		\vfill
	\end{center}
\end{titlepage} 

\newpage

\section*{Cuestionario}
\begin{enumerate}
\item ¿Qué organismos internacionales regulan el comercio internacional? \\
Respuesta: Organización Mundial del Comercio, Organización Mundial de Aduanas, Fondo Monetario Internacional 
\item ¿Cuál es el objetivo del dumping? \\
Respuesta: Ganar mercado en el país importador y perjudicar a competidores locales en el comercio.
\item ¿El desequilibrio comercial ocurre cuando un país exporta mas de lo que importa? \\
Respuesta: Verdadero.
\item ¿A qué se le llama superávit comercial? \\
Respuesta: Cuando hay más exportación y menos importación.
\item Un bloque comercial es un acuerdo entre países para reducir barreras comerciales: \\
	Respuesta: Verdadero.
\item Transacción de bienes y servicios entre un comprador y vendedor. \\
Respuesta: e-commerce.
\item Proceso completo que se lleva a cabo para gestionar un negocio online \\
Respuesta: e-Business.
\item ¿Durante qué década hubo avances tecnológicos que permitieron la llegada del e-commerce? \\
Respuesta: 1990.
\item Se refiere a transacciones entre empresas \\
Respuesta: B2B.
\item Se refiere la venta de productos y servicios directamente a los consumidores finales \\
Respuesta: B2C

\end{enumerate}


\newpage

A continuación se anexa la evidencia del ejercicio realizado en clase:
\vspace{1cm}
\begin{figure}[h!]
\centering
	\includegraphics[width=.5\textwidth]{fig1}
\end{figure}

\end{document}