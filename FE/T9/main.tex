\documentclass[a4paper,12pt]{article}
\usepackage[a4paper, margin=2.5cm]{geometry}
\usepackage[pdftex]{graphicx}
\usepackage{tikz}
\usepackage{pgfplots}
\usepackage{enumitem}
\usepackage{float}
\usepackage[document]{ragged2e}
\usepackage[utf8]{inputenc}
\usepackage[T1]{fontenc}
\usepackage[spanish,es-tabla]{babel}
\renewcommand{\shorthandsspanish}{}
\usepackage{xurl}
\usepackage{lipsum}
\usepackage{mwe}
\usepackage{multicol}
\usepackage{siunitx}
\usepackage{listings}
\usepackage{enumitem}

\usepackage{listings}

\begin{document}

\begin{titlepage}
	\begin{tikzpicture}[overlay, remember picture]
		\path (current page.north east) ++(-0.3,-1.5) node[below left] {\includegraphics[width=0.35\textwidth]{/home/saikkopat/Documents/LOGOS IPN/EscudoESCOM}};
	\end{tikzpicture}
	\begin{tikzpicture}[overlay, remember picture]
		\path (current page.north west) ++(1.5,-1) node[below right] {\includegraphics[width=0.2\textwidth]{/home/saikkopat/Documents/LOGOS IPN/logo}};
	\end{tikzpicture}
	\begin{center}
		\vspace{-1.5cm}
		{\LARGE Instituto Politécnico Nacional\par}
		\vspace{.5cm}
		{\LARGE Escuela Superior de Cómputo\par}
		\vspace{2.5cm}
		{\large Unidad de aprendizaje:}\\{\Large Fundamentos económicos\par}
		\vspace{7cm}
		{\scshape\Huge Actividad 9:\par}
		{\itshape\Large Utilidad total y marginal\par}
		\vfill
		{\Large Alumno: González Cárdenas Ángel Aquilez\par}
		\vspace{1cm}
		{\Large Boleta: 2016630152\par}
		\vspace{1cm}
		{\Large Grupo: 2CV2\par}
		\vspace{1cm}
		{\Large Profesora: Villegas Navarrete Sonia\par}
		\vfill
	\end{center}
\end{titlepage} 

\newpage

\textbf{\Large Ejercicio:}\\
\vspace{0.5cm}

Mariana gasta todo su ingreso en un producto \emph{A}.
Para la resolución del ejercicio designaremos al bien \emph{A} como \textit{Tortas cubanas}.\\
\vspace{.5cm}

\begin{enumerate}[label=\alph*)]

\item Obtenga la gráfica correspondiente a la utilidad total del producto \emph{A}.\par

\item Obtenga la gráfica correspondiente a la utilidad marginal del producto \emph{A}.\par

\end{enumerate}

\vspace{1cm}

De la Tabla 1 se obtienen la $UM_A$ para las \emph{Tortas cubanas} de la fórmula 

\[
	UM = \frac{\Delta UT_x}{\Delta x} = \frac{UT_x2 - UT_x1}{x2-x1}
\]

\begin{table}[ht!]
\begin{center}
\begin{tabular}{|c c c|}
	\hline
		Q & $UT_A$ & $UM_A$\\ [0.5ex]
	\hline
	0 & 0 & -\\
	1 & 6 & 6\\
	2 & 14 & 8\\
	3 & 24 & 10\\
	4 & 28 & 4\\
	5 & 30 & 2\\
	6 & 31 & 1\\
	7 & 31 & 0\\
	8 & 30 & 1\\
	9 & 28 & 2\\	\hline



\end{tabular}
\label{table:1}
\caption{Valores de utilidad total y marginal del bien \emph{A}}
\end{center}
\end{table}


\newpage

\begin{figure}[h!]
	\centering
	
\begin{tikzpicture}

\begin{axis} [
	legend style={legend pos=outer north east},
	width=12cm,	
	grid=both,
	ymin=0,
	xmin=0, xmax=10,
	xtick={0,1,...,10},
	title=Utilidad total y marginal,
	ylabel=Unidades de utilidad total (UUT) y Unidades de utilidad marginal (UUM),
	xlabel=Tortas cubanas,
]

\addplot [color=blue] coordinates {
			(0,0)
			(1,6)
			(2,14)
			(3,24)
			(4,28)
			(5,30)
			(6,31)
			(7,31)
			(8,30)
			(9,28)

};
\addlegendentry{Utilidad total}

\addplot [color=red] coordinates {
			(.5,6)
			(1.5,8)
			(2.5,10)
			(3.5,4)
			(4.5,2)
			(5.5,1)
			(6.5,0)
			(7.5,1)
			(8.5,2)

};
\addlegendentry{Utilidad marginal}

\end{axis}
\end{tikzpicture}
\caption{Utilidad total y marginal}
\end{figure}


\vspace{3cm}
A continuación se anexa el ejercicio realizado en clase:

\end{document}