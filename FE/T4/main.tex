\documentclass[a4paper,12pt]{report}
\usepackage{graphicx}
\usepackage{tikz}
\usepackage{float}
\usepackage[document]{ragged2e}
\usepackage[utf8]{inputenc}
\usepackage[T1]{fontenc}
\usepackage[spanish]{babel}
\renewcommand{\shorthandsspanish}{}
\usepackage{xurl}

\usepackage{listings}

\begin{document}

\begin{titlepage}
	\begin{tikzpicture}[overlay, remember picture]
		\path (current page.north east) ++(-0.3,-1.5) node[below left] {\includegraphics[width=0.4\textwidth]{/home/saikkopat/Documents/LOGOS IPN/EscudoESCOM}};
	\end{tikzpicture}
	\begin{tikzpicture}[overlay, remember picture]
		\path (current page.north west) ++(1.5,-1) node[below right] {\includegraphics[width=0.23\textwidth]{/home/saikkopat/Documents/LOGOS IPN/logo}};
	\end{tikzpicture}
	\begin{center}
		\vspace{-3cm}
		{\LARGE Instituto Politécnico Nacional\par}
		{\Large Escuela Superior de Cómputo\par}
		\vspace{7cm}
		{\scshape\Huge Actividad 4:\par}
		{\itshape\Large Actores económicos\par}
		\vfill
		{\Large Alumno: González Cárdenas Ángel Aquilez\par}
		\vspace{1cm}
		{\Large Boleta: 2016630152\par}
		\vspace{1cm}
		{\Large Grupo: 2CV2\par}
		\vspace{1cm}
		{\Large Unidad de aprendizaje: Fundamentos económicos\par}
		\vspace{1cm}
		{\Large Profesora: Villegas Navarrete Sonia\par}
		\vfill
	\end{center}
\end{titlepage} 

\newpage

\textbf{\Large ¿Qué relación tienen los cuatro actores económicos?}\\
\vspace{0.5cm}

Dentro de la economía, existen cuatro \emph{actores económicos}, siendo el primero de ellos la \textbf{familia}, quienes fundamentalmente poseen la mano de obra necesaria para la realización del trabajo, y son los dueños de la tierra. Previo a la revolución industrial, las familias utilizaban la tierra para abastecerse, y durante el feudalismo comenzaron a desplazarse hacia los talleres, cumpliendo un papel de artesanos. Con la llegada de la industrialización y la modernización, los medios de producción pasaron a manos de los dueños de fabricas, apareciendo el segundo actor económico: la \textbf{empresa}. Con la aparición de la empresa, las familias comenzaron a desplazarse y concentrarse en las ciudades para vender su mano de obra y la tierra, pues al crearse necesidades, el trabajo rudimentario con la tierra no era rival para el los frutos del trabajo producido por una empresa.\par\vspace{1cm}
Si bien la aparición de las empresas modernizo la vida de las personas, comenzó la aparición de sus consecuencias en la vida de las personas, tales como precariedad laboral, jornadas de trabajo extenuantes, vicios como el alcoholismo, etc. Derivado de esto y las protestas de las familias (ahora en calidad de trabajadores asalariados), aparecieron regulaciones para evitar y castigar practicas empresariales que impactaran de forma perjudicial a los trabajadores, entonces el tercer actor económico, el \textbf{gobierno}, comenzó a implementar leyes, normas, regulaciones, sanciones y lineamientos para responder a las demandas de los trabajadores. Estas regulaciones varían de país en país, y la medida en que estas favorecen a la empresa o a los trabajadores corresponde con el nivel de organización de los trabajadores.\par\vspace{1cm}
Finalmente, con el crecimiento que las empresas dominantes de los diferentes mercados, se comenzó con su expansión hacia otros territorios donde no se aprovechaba el mercado de forma efectiva, apareciendo finalmente el cuarto actor económico: el \textbf{sector externo}. Gracias al intercambio de productos producido por exportaciones e importaciones, se aprovechan mejor los diferentes recursos de cada territorio. Cabe señalar que el gobierno, en un intento por aprovechar esta situación, establece impuestos en esta clase de operaciones para ampliar la recaudación y disponer de mayores recursos para los diferentes gastos que considere.\par\vspace{1cm}
En conclusión, las familias trabajan en las empresas quienes responden a las necesidades de las familias con bienes o servicios, donde el gobierno incentiva o castiga las practicas empresariales, y coopera con otros gobiernos y empresas para el intercambio de bienes y servicios (sector externo) a fin de responder a las necesidades que el sector interno no puede solucionar. Así, los diferentes actores económicos se relacionan y responden a los cambios y transformaciones que sufren cada uno, gracias a esto existe una variedad de condiciones óptimas o mejorables para el aprovechamiento de los recursos, que no necesariamente significa una redistribución equitativa de la riqueza.
\end{document}