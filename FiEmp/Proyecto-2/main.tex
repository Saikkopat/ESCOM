\documentclass[a4paper,12pt]{article}
\usepackage[a4paper, margin=2.5cm]{geometry}
\usepackage[pdftex]{graphicx}
\usepackage{tikz}
\usepackage{pgfplots}
\usepackage{enumitem}
\usepackage{float}
\usepackage[document]{ragged2e}
\usepackage[utf8]{inputenc}
\usepackage[T1]{fontenc}
\usepackage[spanish,es-tabla]{babel}
\renewcommand{\shorthandsspanish}{}
\usepackage[nobiblatex]{xurl}
\usepackage{lipsum}
\usepackage{mwe}
\usepackage{multicol}
\usepackage{siunitx}
\usepackage{listings}
\usepackage{enumitem}
\usepackage{amsmath}
\usepackage{listings}
\usepackage{tabularray}
\usepackage{schemata}
\usepackage{hyperref}
\usepackage{booktabs}
\usepackage{pdflscape}
\usepackage{afterpage}


\graphicspath{ {/home/saikkopat/Documents/ESCOM/FiEmp/Proyecto-2} }

\begin{document}

\begin{titlepage}
	\begin{tikzpicture}[overlay, remember picture]
		\path (current page.north east) ++(-0.3,-1.8) node[below left] {\includegraphics[width=0.35\textwidth]{EscudoESCOM}};
	\end{tikzpicture}
	\begin{tikzpicture}[overlay, remember picture]
		\path (current page.north west) ++(1.5,-1) node[below right] {\includegraphics[width=0.2\textwidth]{logo}};
	\end{tikzpicture}
	\begin{center}
		\vspace{-1.5cm}
		{\LARGE Instituto Politécnico Nacional\par}
		\vspace{.5cm}
		{\LARGE Escuela Superior de Cómputo\par}
		\vspace{2.5cm}
		{\large Unidad de aprendizaje:}\\{\Large Finanzas empresariales\par}
		\vspace{5cm}
		{\scshape\Huge Proyecto\par}
		{\itshape\Large Segunda etapa: análisis financiero de dos empresas\par}
		\vfill
		{\Large Integrantes:\par}
		\vspace{0.7cm}
		{\Large Gómez Tovar Yoshua Oziel\par}
		{\Large Boleta: 2023630391\par}
		\vspace{0.5cm}
		{\Large González Cárdenas Ángel Aquilez\par}
		{\Large Boleta: 2016630152\par}
		\vspace{0.5cm}
		{\Large Zarco Sosa Kevin\par}
		{\Large Boleta: 2023630735\par}
		\vspace{1cm}
		{\Large Grupo: 3CV4\par}
		\vspace{1cm}
		{\Large Profesor: Estrada Elizalde Serafín\par}
		\vfill
	\end{center}
\end{titlepage} 

\newpage

\tableofcontents

\newpage


\section{Introducción}

El presente documento detalla la segunda etapa del proyecto perteneciente a la unidad de aprendizaje \emph{Finanzas Empresariales}. En la primera etapa se escogieron a las empresas \emph{Nike} y \emph{Adidas}, para las cuales se analizaron sus estados de resultados.\par
Para esta segunda etapa se recolectó y analizó información a fin de obtener un analisis que permitió construir los respectivos estados proforma de cada empresa.\par

\section{Objetivos}
En el desarrollo del presente documento que continúa con la segunda parte del proyecto se contemplaron los siguientes objetivos de investigación a fin de establecer parametros para la elaboracion de estados proforma de resultados:

\begin{enumerate}
	\item Inversiones de las empresas que hayan realizado o piensen realizar.
	\item Niveles de venta esperados.
	\item Ampliación de sucursales o nuevos mercados.
	\item Innovaciones y desarrollos.
\end{enumerate}

Después, se establecieron supuestos financierons para la elaboracion de estados proforma y el balance general para el año 2024.\par

Finalmente, se procedió con una explicación del analisis realizado.

\clearpage

\section{Investigación de las empresas}

\subsection{Inversiones}

Nike es una de las compañías más innovadoras del mundo, según el ránking elaborado por Boston Consulting Group (BCG) en 2023, donde ocupa el puesto número doce (BCG, 2023). La empresa estadounidense se ha destacado por el desarrollo de calzado de élite que ayuda a los corredores a mejorar su rendimiento, como el modelo Vaporfly, que fue utilizado por el keniano Eliud Kipchoge para batir el récord mundial en el Maratón de Berlín 2023, con un tiempo de 2 horas, 1 minuto y 39 segundos (BBC, 2023). Nike también ha invertido en la Responsabilidad Social Empresarial (RSE) y en el patrocinio de equipos deportivos, deportistas y personalidades influyentes, como el futbolista Cristiano Ronaldo, la tenista Serena Williams o el cantante Drake (Nike, 2023).\par

Gracias a sus campañas publicitarias con personalidades influyentes del deporte y el entretenimiento, Nike ha conseguido duplicar los ingresos en comparación con Adidas para el 2020 según informa el diario \emph{El confidencial}, con un ingreso de $44.538$ millones de dólares en comparación a los $20$ millones reportados por Adidas. En el artículo de \emph{Natalia Mateos}: \textit{Nike vs Adidas: cuando el 'challenger' se hace dueño del mercado con 37.000 M en ventas}, se destacan la importancia de las ventas en línea (\emph{ecommerce}), las campañas de publicidad y se hace enfasis en la ventaja de Nike sobre Adidas en patrocinios deportivos, mismos que le han otorgado una ventaja significativa:

\begin{quotation}
	Nike lidera actualmente la competición con una inversión conjunta en 2021 que superó los 3.000 millones de dólares. Así, la marca del Swoosh superó a Adidas en el pasado ejercicio por 270 millones de dólares. Esto supuso que Nike destinara 1.673 millones de dólares a invertir en jugadores y competiciones, mientras que la compañía alemana destinó 1.026 millones de dólares. 
\end{quotation}

Por otra parte, Adidas ha subido un puesto en el ránking de BCG, situándose en la posición número 35 (BCG, 2023). La empresa alemana ha innovado en la fabricación de calzado deportivo con la tecnología 3D, que permite personalizar los zapatos según las preferencias y necesidades de cada cliente, ofreciendo una mayor comodidad y adaptabilidad (Adidas, 2023). Adidas también ha colaborado con otras grandes marcas de lujo, como Balenciaga y Gucci, para crear líneas streetwear de prêt-à-porter y accesorios que mezclan los códigos icónicos de ambas firmas, generando un gran impacto en el mercado de la moda (Vogue, 2023).\par

Incluso la pandemia por \emph{COVID 19} no detuvo la "rivalidad" de ambas empresas, ya que ambas desarollaron respectivamente cubrebocas, máscaras faciales y "escudos" faciales.\par

En lo que respecta a inversiones a futuro, ambas empresas mantienen estrategias de publicidad permantente, buscando nuevos y más redituables patrocinios y alianzas comerciales. Por otra parte, el aumento en las ventas en linea ha provocado que ambas empresas apuesten por una presencia permantente en internet al apostar por involucarse en las tendencias, como señala (Mateos, 2022):
\begin{quote}
	comenzado ya a dar sus primeros pasos dentro del mundo del metaverso y el blockchain, en un intento más por ampliar la relación con los aficionados y abrirse a las nuevas tendencias del mercado. De este modo, la estadounidense firmó un acuerdo con Roblox para crear un parque temático dentro de su plataforma, mientras que la firma alemana se ha aliado con Bored Ape Yacht Club (Bayc), Punks Comic y Gmone para lanzar un proyecto de NFTs del que todavía no se conocen muchos detalles. 
\end{quote}

Según Vyshnavi y Jain, además de la busqueda de Nike por superar los \$50 millones de dólares en ganancias, espera que para el 2025 la mitad de sus ingresos sean por operaciones digitales. Por otra parte, también se indica que Adidas busca conseguir una produccion con cero emisiones de gases de efecto invernadero en una escala global para el 2050. Por su parte, Nike indica que el 78\% de sus productos ya contienen materiales sostenibles y busca seguir aumentando esta cifra hasta alcanzar el 100\%. \par


\subsection{Niveles de venta esperados}

\begin{enumerate}
	\item Nike: La empresa estadounidense prevé alcanzar unos ingresos de 54.000 millones de dólares en 2024, lo que supondría un crecimiento del 5.4\% respecto al año fiscal 2023, cuando facturó 51.217 millones de dólares.
	\item Adidas: La empresa alemana espera obtener unos ingresos de 27.272 millones de dólares en 2024, lo que representaría un aumento del 17.7\% en comparación con el año 2021, cuando ingresó 23.161 millones de dólares.

\end{enumerate}

Nike se basa en su estrategia de innovación, digitalización y expansión internacional para lograr sus objetivos, mientras que Adidas confía en su capacidad de adaptación al mercado, su diversificación de productos y su apuesta por la sostenibilidad para impulsar su crecimiento. \par

\subsection{Ampliación de sucursales o nuevos mercados}

Previamente en la primera etapa, se analizó e identifico el crecimiento de ambas empresas hacia los mercados asiaticos y latinoamericanos, tanto en ventas como en manufactura de productos.\par

Aquí hay una investigación sobre la ampliación de sucursales o tiendas de Nike y Adidas, y su expansión hacia nuevos mercados, basada en los resultados de la búsqueda web:\par
Nike ha iniciado un cambio de estrategia más centrada en apostar por las tiendas propias de la marca para la venta de sus productos, dejando de lado el modelo multimarca. Sin embargo, esta nueva línea de negocio se ha visto afectada por la pandemia de COVID-19, que ha provocado el cierre de muchas tiendas y la caída de las ventas. Nike ha reaccionado impulsando su negocio online, que ha crecido un 82\% en el primer trimestre fiscal de 2021. Además, Nike ha seguido expandiéndose en mercados emergentes como China, donde ha aumentado sus ingresos un 6\% en el mismo periodo.\par
Por otra parte, Adidas también ha sufrido el impacto de la crisis sanitaria, que ha reducido sus ventas un 16\% y sus beneficios un 78\% en 2020. Adidas ha anunciado la venta de Reebok, la marca que compró en 2006 para competir con Nike, pero que no ha logrado el rendimiento esperado. Adidas se ha enfocado en crecer sus ventas online, que han subido un 53\% en 2020 y que espera duplicar para 2025. Asimismo, Adidas ha apostado por la sostenibilidad y la innovación, lanzando productos con materiales reciclados o biodegradables, como los tenis hechos con hongos. Adidas también ha buscado ampliar su presencia en mercados como China, América Latina y Asia. \par

\subsection{Innovaciones y desarrollos}

Ambas empresas han desarrollado e innovado en el campo del calzado deportivo, ofreciendo productos de alta calidad y rendimiento. Algunos de los desarrollos e innovaciones que han presentado son:

\emph{Nike:}

\begin{enumerate}
	\item Tecnología Flyknit: Nike ha sido líder en la introducción de tecnologías innovadoras en sus productos. La tecnología Flyknit, por ejemplo, se introdujo para crear calzado ligero y transpirable mediante la tejeduría de hilos sintéticos. Esta tecnología se inspiró en los comentarios de los corredores, que buscaban un calzado que se ajustara como un calcetín, con el soporte y la durabilidad necesarios para practicar deporte. Algunos de los modelos que utilizan Flyknit son el Nike Air VaporMax Flyknit, el Nike Epic React Flyknit y el Nike Flyknit Racer.
	\item Nike React: La tecnología de amortiguación Nike React es conocida por su capacidad de respuesta y comodidad. Se utiliza en una variedad de calzado deportivo para mejorar la experiencia del usuario. Según Nike, React es una espuma que combina polímeros para lograr una sensación de ligereza, suavidad, durabilidad y elasticidad. Algunos de los modelos que utilizan React son el Nike React Infinity Run, el Nike React Vision y el Nike React Element 87.
	\item Nike Adapt: Nike ha avanzado en el mundo de la tecnología de calzado con la introducción de Nike Adapt, que permite a los usuarios ajustar el ajuste de su calzado a través de una aplicación móvil. Esta tecnología se basa en el sistema de atado inteligente que debutó en el Nike HyperAdapt 1.0 en 2016, que se ajustaba automáticamente al pie del usuario. Algunos de los modelos que utilizan Adapt son el Nike Adapt BB, el Nike Adapt Auto Max y el Nike Adapt Huarache.
\end{enumerate}

Asimismo, Nike ha estado trabajando en iniciativas sostenibles, como la utilización de materiales reciclados y la reducción de residuos en la fabricación de productos. Por ejemplo, Nike colabora con Parley for the Oceans para crear productos con plásticos reciclados recogidos del océano, como el Nike Air VaporMax 2020 Flyknit. También ha lanzado la colección Nike Space Hippie, que utiliza materiales reciclados y desechos de fábrica para crear calzado con un bajo impacto ambiental.

\emph{Adidas:}
\begin{enumerate}
	\item Boost Technology: Adidas introdujo la tecnología Boost en su línea de calzado, que se centra en proporcionar un retorno de energía superior y una amortiguación excepcional. Según Adidas, Boost es una espuma que se compone de miles de partículas de TPU expandido, que absorben y liberan la energía de cada pisada. Algunos de los modelos que utilizan Boost son Adidas Ultraboost, Adidas NMD y Adidas Yeezy Boost.
	\item Futurecraft: Adidas ha presentado la serie Futurecraft, que busca explorar nuevas formas de fabricación de calzado, incluyendo la impresión 3D y otros métodos innovadores. Según Adidas, Futurecraft es una iniciativa que combina arte, diseño, ciencia y colaboración para crear productos que se adapten a las necesidades de cada atleta. Algunos de los modelos que forman parte de Futurecraft son Adidas Futurecraft 4D, Adidas Futurecraft Loop y Adidas Futurecraft Strung.
	\item Adidas Parley: En colaboración con Parley for the Oceans, Adidas ha lanzado productos fabricados con plásticos reciclados recogidos del océano, destacando su compromiso con la sostenibilidad ambiental. Según Adidas, cada par de calzado Parley evita que aproximadamente 11 botellas de plástico lleguen al océano. Algunos de los modelos que utilizan Parley son Adidas Ultraboost x Parley, Adidas Terrex Two Parley y Adidas 4DFWD x Parley.
	\item Calzado personalizado: Adidas ha incursionado en la personalización de productos a través de la iniciativa miadidas, permitiendo a los clientes diseñar su propio calzado con colores y detalles específicos. Según Adidas, miadidas ofrece la posibilidad de crear calzado único y original, que reflejen el estilo personal de cada usuario. Algunos de los modelos que se pueden personalizar son Adidas Superstar, Adidas Stan Smith y Adidas ZX Flux.
	\item Adidas 4D: La tecnología de impresión 4D de Adidas busca crear entresuelas personalizadas mediante la impresión 3D, brindando una amortiguación adaptativa basada en las necesidades individuales del usuario. Según Adidas, 4D es una tecnología que utiliza la luz, el oxígeno y el líquido para crear una estructura de rejilla que proporciona un soporte preciso y un retorno de energía óptimo. Algunos de los modelos que utilizan 4D son Adidas Ultra 4D, Adidas 4DFWD y Adidas 4D Run 1.0.

\end{enumerate}

\clearpage
\newpage

\section{Estados proforma y estado de resultados}

A continuacion se muestran los estados proforma realizados para las dos empresas, a partir de la información recolectada y supuestos financieron a fin de realizar el ejercicio financiero correspondiente al 2024 considerando las ventas esperadas por ambas empresas.\par

\subsection{Nike}


\begin{table}[h]
\centering
\caption{Estado de resultados de Nike en los años fiscales 2022, 2023 y 2024 (proforma)}
\vspace{1cm}
\begin{tabular}{lrrr}
\toprule
 & \textbf{2022} & \textbf{2023} & \textbf{2024 (proforma)} \\
\midrule
Ingresos & $44,500$ & $51,217$ & $57,516$ \\
Costo de ventas & $24,600$ & $27,900$ & $31,188$ \\
\midrule
Margen bruto & $19,900$ & $23,317$ & $26,328$ \\
Gastos de venta y administración & $13,400$ & $14,800$ & $16,576$ \\
Gastos de investigación y desarrollo & $400$ & $500$ & $600$ \\
Otros gastos & $200$ & $300$ & $360$ \\
\midrule
Ganancia de las operaciones & $5,900$ & $7,717$ & $8,792$ \\
Gastos financieros & $300$ & $400$ & $480$ \\
Ingresos financieros & $100$ & $200$ & $240$ \\
\midrule
Ganancia antes de impuestos & $5,700$ & $7,517$ & $8,552$ \\
Impuestos sobre la renta & $1,400$ & $1,800$ & $2,060$ \\
\midrule
Ganancia neta & $4,300$ & $5,717$ & $6,492$ \\
Ganancia por acción básica & $2.80$ & $3.72$ & $4.20$ \\
Ganancia por acción diluida & $2.77$ & $3.69$ & $4.18$ \\
\bottomrule
\end{tabular}
\end{table}


\clearpage
\newpage


\afterpage{%
  \clearpage
  \begin{landscape}
    \begin{table}[h]
    \centering
    \caption{Balance general proforma de Nike para los años fiscales 2023 y 2024}
	 \vspace{.5cm}
    \begin{tabular}{lrrrr}
    \toprule
     & \textbf{Año fiscal 2022} & \textbf{Año fiscal 2023 (proforma)} & \textbf{Año fiscal 2024 (proforma)} \\
    \midrule
    \textbf{Activos corrientes} & 27,447 & 31,564.05 & 35,338.49 \\
    Efectivo e inversiones a corto plazo & 10,621 & 12,215.15 & 13,649.72 \\
    Cuentas por cobrar, Neto & 5,437 & 6,249.55 & 7,349.48 \\
    Inventarios, Neto & 9,326 & 10,724.90 & 12,005.87 \\
    Otros activos corrientes, Neto & 765 & 879.75 & 983.67 \\
    \midrule
    \textbf{Activos no corrientes} & 12,567 & 14,451.05 & 16,202.07 \\
    Inmuebles, mobiliario y equipo (neto) & 7,663 & 8,818.45 & 9,899.01 \\
    Intangibles, Neto & 280 & 322.00 & 359.84 \\
    Otros activos permanentes, Total & 194 & 223.10 & 249.51 \\
    \midrule
    \textbf{Total activos} & \textbf{39,647} & \textbf{45,015.10} & \textbf{51,540.56} \\
    \midrule
    \textbf{Pasivos corrientes} & 10,199 & 11,719.85 & 13,135.04 \\
    Cuentas por pagar & 2,810 & 3,237.50 & 3,620.40 \\
    Inversiones a corto plazo & 4,243 & 4,879.55 & 5,461.46 \\
    Documentos por pagar/Deuda a corto plazo & 7 & 8.05 & 8.99 \\
    Deudas con entidades de crédito y obligaciones & 926 & 1,065.90 & 1,192.66 \\
    \midrule
    \textbf{Total pasivo} & \textbf{24,375} & \textbf{28,058.30} & \textbf{31,418.41} \\
    \midrule
    \textbf{Capital contable} & 15,272 & 16,956.80 & 20,122.15 \\
    Acciones comunes Total & 3 & 3 & 3 \\
    Prima en venta de acciones & 11,851 & 13,634.65 & 15,273.07 \\
    Resultado de ejercicios anteriores & 2,859 & 3,288.15 & 3,679.64 \\
    Otras participaciones, Total & 559 & 642.00 & 718.44 \\
    \midrule
    \textbf{Total Pasivo y Capital Contable} & \textbf{39,647} & \textbf{45,015.10} & \textbf{51,540.56} \\
    \bottomrule
    \end{tabular}
    \end{table}
  \end{landscape}
}

\clearpage
\newpage

\begin{landscape}
\begin{table}[h]
\centering
\caption{Cálculos adicionales}
\vspace{1cm}
\begin{tabular}{lr}
\toprule
\textbf{Cálculo} & \textbf{Valor} \\
\midrule
Ingresos proyectados 2024 & $51,217 \times (1 + 0.12) = 57,516$ \\
Costo de Ventas proyectado 2024 & $57,516 \times 0.60 = 34,509.6$ \\
Suponiendo que el costo de ventas se mantiene en el 60 porciento de los ingresos\\
Gastos de venta y administración proyectados 2024 & $57,516 \times 0.30 = 17,254.8$ \\
Suponiendo que los gastos de venta y administración representan el 30 porciento de los ingresos\\, los gastos de investigación y desarrollo el 1 porciento y otros gastos el 0.62 porciento:\\
Gastos de investigación y desarrollo proyectados 2024 & $57,516 \times 0.01 = 575.16$ \\
Otros gastos proyectados 2024 & $57,516 \times 0.0062 = 356.4912$ \\
\bottomrule
\end{tabular}
\end{table}
\end{landscape}


\clearpage

\newpage

\subsection{Adidas}

\begin{table}[ht!]
    \centering
    \begin{tabular}{c c c c}
        \hline
         &  \textbf{2022} &  \textbf{2023} &  \textbf{2024 (Proforma)} \\
        \hline
        Ventas, Netas & 22,511 & 25,887.65 & 30,288.55 \\
        \hline
        Costo de Ventas & 11,817 & 13,589.55 & 15,900 \\
        \hline
        Utilidad Bruta & 10,694 & 12,298.1 & 14,388.55 \\
        \hline
         \textbf{Gastos Opertaivos} &  &  &  \\
        \hline
        Venta y Administración General \\Mantenimiento/ Renta de equipo, Total & 10,078 & 11,589.7 & 13,559.94 \\
        \hline
        Intereses Pagados (Utilidad), Neto & -137 & -157.55 & -184.33 \\
        \hline
        Otros gastos de Operación, Total & 21 & 24.15 & 28.2 \\
        \hline
        Utilidad Operativa & 732 & 841.8 & 984.74 \\
        \hline
        Total de Impuestos a la Utilidad & 134 & 154 & 180.1 \\
        \hline
         \textbf{Utilidad Neta} & 598 & 687.7 & 804.64 \\
        \hline
    \end{tabular}
    \caption{Estado de Resultados Adidas}
    \label{tab:tabla18x4}
\end{table}



\clearpage
\newpage

\afterpage{%
  \clearpage
  \begin{landscape}
    \begin{table}[h]
    \centering
    \caption{Balance general proforma de Adidas para los años fiscales 2023 y 2024}
    \begin{tabular}{lrrrr}
    \toprule
     & \textbf{Año fiscal 2022} & \textbf{Año fiscal 2023 (proforma)} & \textbf{Año fiscal 2024 (proforma)} \\
    \midrule
    \textbf{Activos corrientes} &  &  &  \\
    Efectivo & 966,000 & 1,110,900 & 1,299,753 \\
    Cuentas Pendientes & 2,529,000 & 2,908,350 & 3,402,769.5 \\
    Inventarios & 5,973,000 & 6,868,950 & 8,036,671.5 \\
    Otros Activos Corrientes & 419,000 & 481,850 & 563,764.5 \\
    \midrule
    \textbf{Activos No Corrientes} &  &  &  \\
    Activos Fijos & 4,943,000 & 5,684,450 & 6,650,806.5 \\
    Fondo de Comercio & 1,260,000 & 1,449,000 & 1,695,330 \\
    Intangibles & 429,000 & 493,350 & 577,219.5 \\
    Otros Activos a Largo Plazo & 4,000 & 4,600 & 5,382 \\
    \midrule
    \textbf{Total Activos} & \textbf{20,296,000} & \textbf{23,384,400} & \textbf{27,359,748} \\
    \midrule
    \textbf{Pasivos Corrientes} &  &  &  \\
    Deuda Corriente & 527,000 & 606,050 & 709,079 \\
    Cuentas a Pagar & 2,908,000 & 3,344,200 & 3,912,714 \\
    Otro Pasivo Corriente & 72,000 & 82,800 & 95,940 \\
    \midrule
    \textbf{Pasivos No Corrientes} &  &  &  \\
    Deuda a Largo Plazo & 2,946,000 & 3,387,000 & 3,962,790 \\
    Pasivo por Impuesto Diferido & 135,000 & 155,250 & 181,643 \\
    Otro Pasivo Corriente & 72,000 & 82,800 & 95,940 \\
    Otro Pasivo a Largo Plazo & 1,000 & 1,150 & 1,343.5 \\
    \midrule
    \textbf{Total Pasivo} & \textbf{14,945,000} & \textbf{17,186,750} & \textbf{20,108,497.5} \\
    \midrule
    \textbf{Capital Contable} & 5,351,000 & 6,197,650 & 7,251,250.5 \\
    \midrule
    \textbf{Total Pasivo y Capital Contable} & \textbf{20,296,000} & \textbf{23,384,400} & \textbf{27,359,748} \\
    \bottomrule
    \end{tabular}
    \end{table}
  \end{landscape}
}


\clearpage
\newpage


\section{Comentarios sobre los estados de resultados}

\subsection{Nike}

Primero, los puntos destacables del Balance general del 2022 al 2023 son:
\begin{enumerate}
	\item Crecimiento Sostenible: El aumento uniforme en activos y pasivos sugiere un crecimiento sostenible y planificado.
	\item Gestión de Deuda: El incremento en la deuda a largo plazo puede indicar una estrategia de financiamiento para respaldar las iniciativas de expansión.
	\item Inversiones en Activos Fijos: El aumento en inmuebles, mobiliario y equipo sugiere inversiones significativas en activos productivos.
	\item Posicionamiento Financiero Mejorado: En conjunto, los cambios indican un fortalecimiento del posicionamiento financiero de Nike, preparando el terreno para el año fiscal 2024.
\end{enumerate}

Después, sobre el Estado Proforma de Nike para el Año Fiscal 2023 tenemos que:
\begin{enumerate}
	\item Crecimiento Sostenible: Se evidencia un aumento significativo del 15\% en los activos corrientes, reflejando un crecimiento sostenible y una posible expansión en las operaciones comerciales.
	\item Foco en Liquidez: El incremento del 15\% en efectivo e inversiones a corto plazo indica un enfoque estratégico en la liquidez, proporcionando a la empresa flexibilidad financiera.
	\item Expansión en Cuentas por Cobrar: El aumento del 15\% en cuentas por cobrar sugiere una estrategia de expansión de ventas a crédito para impulsar el crecimiento de los clientes y fortalecer las relaciones comerciales.
	\item Inversiones en Activos No Corrientes:El incremento del 15\% en activos no corrientes, como propiedades y equipos, sugiere inversiones en infraestructura para respaldar el crecimiento a largo plazo.
	\item Gestión Prudente de Deuda: Aunque no se especifica la deuda, el crecimiento general podría implicar una gestión prudente de la deuda para financiar estratégicamente proyectos y expansiones.
\end{enumerate}

Asi, para la Proyección Proforma de Nike para el Año Fiscal 2024 tenemos que:
    \begin{enumerate}
		\item Proyección Moderada: La proyección de un aumento del 12\% en activos indica una perspectiva más moderada para el crecimiento en comparación con el año anterior, posiblemente considerando factores económicos y de mercado.
		\item Continuación del Enfoque en Liquidez: La mantención de un aumento en efectivo e inversiones a corto plazo sugiere la continuación del enfoque en la liquidez para afrontar posibles desafíos o aprovechar oportunidades.
		\item Gestión Equilibrada de Activos No Corrientes: El aumento del 12\% en activos no corrientes indica una gestión equilibrada, posiblemente priorizando inversiones estratégicas y manteniendo un crecimiento sostenible.
		\item Control en Cuentas por Cobrar: El aumento del 12\% en cuentas por cobrar podría sugerir un enfoque más controlado en las ventas a crédito, quizás en respuesta a lecciones aprendidas o cambios en el entorno empresarial.
		\item Continuidad en Estrategias Financieras: La gestión equilibrada de activos y la proyección conservadora podrían indicar una continuidad en las estrategias financieras de Nike, priorizando estabilidad y rentabilidad.
	 \end{enumerate}

Supuesto financiero: aumento de 12\% en ventas para la proyección de 2024 (Estado Proforma):
\begin{enumerate}
	\item Expectativas Conservadoras: La proyección del 12\% puede ser una estimación más conservadora y realista, teniendo en cuenta factores como la incertidumbre del mercado, posibles cambios económicos y variaciones estacionales.
	\item Ciclos de Negocio: Las industrias, incluida la de artículos deportivos como Nike, pueden experimentar ciclos de negocios. La empresa podría anticipar una desaceleración natural en el crecimiento después de un año particularmente fuerte.
	\item Competencia y Mercado: Cambios en la dinámica competitiva o en el mercado en general podrían influir en la proyección. Si Nike prevé una mayor competencia o cambios en la demanda del mercado, podría reflejar una proyección más moderada.
	\item Inversiones en Infraestructura: Si Nike está planificando inversiones significativas en infraestructura, expansión geográfica o desarrollo de productos, es posible que haya considerado un crecimiento más lento en el corto plazo para facilitar estas inversiones.
	\item Factores Económicos Globales: Eventos macroeconómicos a nivel mundial pueden tener un impacto en las proyecciones. Consideraciones como tasas de interés, políticas comerciales globales o crisis económicas pueden influir en las decisiones estratégicas.
	\item Estabilidad Financiera: Nike podría estar priorizando la estabilidad financiera y la gestión prudente del crecimiento, evitando proyecciones excesivamente optimistas que podrían llevar a riesgos financieros no deseados.
\end{enumerate}


\subsection{Adidas}

El crecimiento de Adidas es muy similar al de Nike, a excepción de la proyección en el año 2024 es de un porcentaje ligeramente mayor al de Nike.\\
Ambas empresas al ser de giros similares, tener estrategias y magnitudes de ventas muy grandes, pueden mostrar similitudes en sus crecimientos.

Si bien, un porcentaje de crecimiento puede ser mayor para Adidas, en magnitud el crecimiento es menor, ya que por una parte Nike espera poder rebasar ingresos por 50 mil millones de dólares, mientras que Adidas planea rebasar los 27 mil millones.


\clearpage
\newpage

\section{Referencias}

\begin{enumerate}
	\item ¿Cómo puedo aprovechar al máximo mi Nike Adapt? (s/f). Nike.com. Recuperado el 24 de noviembre de 2023, de \url{https://www.nike.com/mx/help/a/adapt}
	\item Mateos, N. (2022, marzo 16). Nike vs Adidas: cuando el "challenger" se hace dueño del mercado con 37000 millones en ventas. El Confidencial. Recuperado de \url{https://www.elconfidencial.com/empresas/2022-03-16/nike-adidas-challenger-dueno-mercado-ventas-deportes_3392667/}
	\item Govind, A. (2023, abril 11). Nike's digital ecosystem outlook - Ashane Govind. Medium. Recuperado de \url{https://ashanegovind.medium.com/nikes-digital-ecosystem-outlook-11fc791b70c1}
	\item Nike Flyknit. (s/f). Nike.com. Recuperado el 24 de noviembre de 2023, de \url{https://www.nike.com/mx/flyknit}
	\item Nike FlYKNIT; ¿Qué es y cuales son sus beneficios? (s/f). SPORTLAND MX. Recuperado el 24 de noviembre de 2023, de \url{https://sportlandmx.com/blogs/noticias/nike-flyknit-qu-es-y-cuales-son-sus-beneficios}
	\item Nike React shoes. (s/f). Nike.com. Recuperado el 24 de noviembre de 2023, de \url{https://www.nike.com/w/react-shoes-7cmrozy7ok}
	\item Nike rompe barreras en el entorno social y ambiental. (2022, abril 22). Greentology. \url{https://greentology.life/2022/04/22/nike-rompe-barreras-en-el-entorno-social-y-ambiental/}
	\item Vyshnavi, P. V. (2023, febrero 6). Nike vs Adidas: Who is Leading The Market? StartupTalky. \url{https://startuptalky.com/adidas-vs-nike/}
	\item (S/f-a). Adidas-group.com. Recuperado el 24 de noviembre de 2023, de \url{https://www.adidas-group.com/en/about/history/}
	\item (S/f-b). Adidas.com. Recuperado el 24 de noviembre de 2023, de \url{https://www.adidas.com/us/blog/439874-what-is-adidas-boost-technology}
	\item (S/f-c). Adidas.com. Recuperado el 24 de noviembre de 2023, de \url{https://www.adidas.com/us/parley}
	\item (S/f-d). Adidas.mx. Recuperado el 24 de noviembre de 2023, de \url{https://www.adidas.mx/personalizado}
	\item Parker, A. B. (2022, mayo 23). The Balenciaga and Adidas collaboration is here—and already selling out. Vogue. \url{https://www.vogue.com/article/balenciaga-and-adidas-collaboration}
	\item Manly, J., Ringel, M., MacDougall, A., Cornock, W., Harnoss, J. D., Baeza, R., Kimura, R., Ward, M., Viner, B., Izaret, J.-M., Backler, W., Lukic, V., Duranton, S., y de Laubier, R. (2023, mayo 23). Most innovative companies 2023: Reaching new heights in uncertain times. BCG Global. \url{https://www.bcg.com/publications/2023/advantages-through-innovation-in-uncertain-times}


\end{enumerate}



\end{document}