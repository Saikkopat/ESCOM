\documentclass[a4paper,12pt]{article}
\usepackage[a4paper, margin=2cm]{geometry}
\usepackage[pdftex]{graphicx}
\usepackage{tikz}
\usepackage{pgfplots}
\usepackage{enumitem}
\usepackage{float}
\usepackage[document]{ragged2e}
\usepackage[utf8]{inputenc}
\usepackage[T1]{fontenc}
\usepackage[spanish,es-tabla]{babel}
\renewcommand{\shorthandsspanish}{}
\usepackage{xurl}
\usepackage{lipsum}
\usepackage{mwe}
\usepackage{multicol}
\usepackage{siunitx}
\usepackage{listings}
\usepackage{enumitem}
\usepackage{amsmath}
\usepackage{listings}
\usepackage{tabularray}

\usepackage{schemata}
\usepackage{pdflscape}
\usepackage{fancyhdr}

\def\fillandplacepagenumber{%
 \par\pagestyle{empty}%
 \vbox to 0pt{\vss}\vfill
 \vbox to 0pt{\baselineskip0pt
   \hbox to\linewidth{\hss}%
   \baselineskip\footskip
   \hbox to\linewidth{%
     \hfil\thepage\hfil}\vss}}


\begin{document}

\begin{titlepage}
	\begin{tikzpicture}[overlay, remember picture]
		\path (current page.north east) ++(-0.15,-1.6) node[below left] {\includegraphics[width=0.35\textwidth]{/home/saikkopat/Documents/LOGOS IPN/EscudoESCOM}};
	\end{tikzpicture}
	\begin{tikzpicture}[overlay, remember picture]
		\path (current page.north west) ++(1.5,-1) node[below right] {\includegraphics[width=0.2\textwidth]{/home/saikkopat/Documents/LOGOS IPN/logo}};
	\end{tikzpicture}
	\begin{center}
		\vspace{-1.5cm}
		{\LARGE Instituto Politécnico Nacional\par}
		\vspace{.5cm}
		{\LARGE Escuela Superior de Cómputo\par}
		\vspace{2.5cm}
		{\large Unidad de aprendizaje:}\\{\Large Finanzas empresariales\par}
		\vspace{5cm}
		{\scshape\Huge Actividad 3:\par}
		{\itshape\Large Entorno de la empresa\par}
		\vfill
		{\Large Integrantes:\par}
		\vspace{0.7cm}
		{\Large Gómez Tovar Yoshua Oziel\par}
		{\Large Boleta: 2023630391\par}
		\vspace{0.5cm}
		{\Large González Cárdenas Ángel Aquilez\par}
		{\Large Boleta: 2016630152\par}
		\vspace{0.5cm}
		{\Large Zarco Sosa Kevin\par}
		{\Large Boleta: 2023630735\par}
		\vspace{1cm}
		{\Large Grupo: 3CV4\par}
		\vspace{1cm}
		{\Large Profesor: Estrada Elizalde Serafín\par}
		\vfill
	\end{center}
\end{titlepage} 

\newpage

\begin{landscape}
\section*{Entorno de la empresa}

De acuerdo a la lectura elabora un cuadro sinóptico para representar el entorno macroeconómico de la empresa y sus aspectos, con una breve definición. De igual forma otro cuadro sinóptico para expresar el entorno
microeconómico.\par

\vspace{0.2cm}

\begin{figure}[ht!]
	\schema {\schemabox{Entorno\\macroeconómico}} 
		{
			\schema {\schemabox{Condiciones\\macroeconómicas}}
				{
					\schema {\schemabox{Producto Interno Bruto (PIB)}}
						{
							{\schemabox{Medida del valor de todos los\\bienes y servicios producidos en una economía.}}
						}
					
					\schema {\schemabox{Inflación}}
						{
							{\schemabox{Aumento sostenido de los precios de bienes y servicios en una economía.}}
						}
					
					\schema {\schemabox{Empleo}}
						{
							{\schemabox{Indicador de la tasa de desempleo y la disponibilidad de mano de obra.}}
						}
					
					\schema {\schemabox{Política monetaria\\y fiscal}}
						{
							{\schemabox{Control de la oferta de dinero y regulación de gastos e\\impuestos por parte del gobierno.}}
						}
					
					\schema {\schemabox{Ciclo económico}}
						{
							{\schemabox{Patrón de fluctuación que experimenta una economía a lo largo \\del tiempo: Expansión, pico, contracción, fondo y recuperación.}}
						}
				}
				
			\schema {\schemabox{Contexto\\socioeconómico}}
				{
					\schema {\schemabox{Demografía}}
						{
							{\schemabox{Características de la población, como edad, migración, y grupos étnicos.}}
						}
					
					\schema {\schemabox{Condiciones económicas}}
						{
							{\schemabox{Poder adquisitivo, sectores de inversión y capacidad de\\ahorro de la población.}}
						}
					
					\schema {\schemabox{Aspecto legal}}
						{
							{\schemabox{Normativas, leyes y regulaciones que afectan a las empresas en un área.}}
						}
					
					\schema {\schemabox{Características\\culturales y sociales}}
						{
							{\schemabox{Tradiciones, costumbres y comportamientos sociales en una sociedad.}}
						}
					
					\schema {\schemabox{Situación política}}
						{
							{\schemabox{Condiciones y estabilidad del entorno político en una región.}}
						}
					
					\schema {\schemabox{Recursos tecnológicos}}
						{
							{\schemabox{Infraestructuras, tecnologías disponibles y conectividad en la sociedad.}}
						}
					
					\schema {\schemabox{Medioambiente}}
						{
							{\schemabox{Consideraciones sobre la conservación de recursos naturales y el\\impacto ambiental.}}
						}
				}
		}

\end{figure}
\fillandplacepagenumber
\end{landscape}

\begin{landscape}


\begin{figure}[ht!]
\vspace{2.5cm}
	\schema {\schemabox{Entorno\\microeconómico}} 
		{
			\schema {\schemabox{Competidores}}
				{
					{\schemabox{Empresas que operan en el mismo mercado y ofrecen productos o servicios similares.}}
				}
			\vspace{0.5cm}
			\schema {\schemabox{Clientes}}
				{
					{\schemabox{Personas o empresas que compran los productos o servicios de la empresa.}}
				}
			\vspace{0.5cm}
			\schema {\schemabox{Proveedores}}
				{
					{\schemabox{Empresas o individuos que suministran materias primas, insumos o servicios a la empresa.}}
				}
			\vspace{0.5cm}
			\schema {\schemabox{Público}}
				{
					{\schemabox{Los consumidores, la comunidad y otros grupos de interés que pueden afectar o ser afectados por la empresa.}}
				}
			\vspace{0.5cm}
			\schema {\schemabox{Distribuidores}}
				{
					{\schemabox{Empresas o intermediarios que ayudan a llevar los productos de la empresa al mercado.}}
				}
			\vspace{0.5cm}
			\schema {\schemabox{Trabajadores}}
				{
					{\schemabox{Los empleados de la empresa y sus habilidades, actitudes y productividad.}}
				}
			\vspace{0.5cm}
			\schema {\schemabox{Demanda}}
				{
					{\schemabox{La cantidad de productos o servicios que los clientes desean y están dispuestos a comprar.}}
				}
			\vspace{0.5cm}
			\schema {\schemabox{Calidad}}
				{
					{\schemabox{La capacidad de los proveedores para proporcionar los insumos necesarios y su estado óptimo para su uso.}}
				}
			\vspace{0.5cm}
			\schema {\schemabox{Cadena de distribución}}
				{
					{\schemabox{La red de intermediarios y métodos utilizados para llevar los productos al mercado y al cliente.}}
				}

		}

\end{figure}
\fillandplacepagenumber
\end{landscape}

\section*{Conclusiones de equipo}

\vspace{1cm}

\begin{quotation}
	La planeación financiera es una parte crucial de la gestión empresarial que involucra la proyección y administración de los recursos financieros de una empresa. Sin embargo, no tener en cuenta los entornos macroeconómicos y microeconómicos puede tener graves consecuencias en la salud financiera y el éxito a largo plazo de una organización.\par

	\vspace{0.5cm}

	Primero, uno de los principales problemas de no considerar los entornos macroeconómicos en la planeación financiera es la vulnerabilidad a los cambios económicos externos. Si una empresa no monitorea las tendencias macroeconómicas, como el PIB, la inflación o las tasas de interés, puede verse sorprendida por situaciones adversas. Por ejemplo, una recesión económica puede afectar negativamente la demanda de productos o servicios, lo que podría llevar a una disminución en los ingresos y beneficios. Esta falta de anticipación puede hacer que la empresa no esté preparada para tomar medidas proactivas, como reducir costos o ajustar su estrategia de precios.\par

	\vspace{0.5cm}

	Por otra parte, el entorno microeconómico de una empresa, que incluye factores como la competencia, los clientes y los proveedores, también es crítico en la planeación financiera. No evaluar adecuadamente estos factores puede llevar a una pérdida de competitividad en el mercado. Por ejemplo, si una empresa ignora a sus competidores y no se adapta a los cambios en las preferencias del cliente, podría perder cuota de mercado. Además, no considerar a los proveedores y la calidad de los insumos puede afectar la eficiencia y la calidad de los productos o servicios ofrecidos. En última instancia, esto podría llevar a una disminución de los ingresos y la rentabilidad.\par

	\vspace{0.5cm}

	Así, la falta de consideración de los entornos macroeconómicos y microeconómicos en la planeación financiera puede llevar a una falta de sostenibilidad y adaptación a largo plazo. Las empresas que no están atentas a las tendencias y cambios económicos pueden quedarse atrás en un entorno empresarial en constante evolución. Esto puede resultar en una pérdida de oportunidades y en la incapacidad de ajustarse a nuevas condiciones del mercado. La adaptación es esencial para la supervivencia y el crecimiento continuo de una empresa, y no considerar estos entornos puede poner en peligro esa capacidad de adaptación.\par

	\vspace{0.5cm}

	En resumen, la falta de valoración de los entornos macroeconómicos y microeconómicos en la planeación financiera puede tener graves consecuencias para una empresa. Estas incluyen vulnerabilidad a cambios económicos, riesgo de pérdida de competitividad y una falta de sostenibilidad a largo plazo. Es esencial que las empresas integren un análisis detenido de estos entornos en su proceso de toma de decisiones financieras para mantenerse competitivas y resilientes en un mundo empresarial dinámico y desafiante.
\end{quotation}


\end{document}